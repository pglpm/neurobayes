\pdfoutput=1
%% Author: PGL  Porta Mana
%% Created: 2019-02-16T18:38:32+0100
%% Last-Updated: 2019-04-03T16:55:04+0200
%%%%%%%%%%%%%%%%%%%%%%%%%%%%%%%%%%%%%%%%%%%%%%%%%%%%%%%%%%%%%%%%%%%%%%%%%%%%
\newif\ifarxiv
\arxivfalse
\ifarxiv\pdfmapfile{+classico.map}\fi
\newif\ifafour
\afourfalse% true = A4, false = A5
\newif\iftypodisclaim % typographical disclaim on the side
\typodisclaimtrue
\newcommand*{\memfontfamily}{zplx}
\newcommand*{\memfontpack}{newpxtext}
\documentclass[\ifafour a4paper,12pt,\else a5paper,10pt,\fi%extrafontsizes,%
onecolumn,oneside,article,%french,italian,german,swedish,latin,
british%
]{memoir}
\newcommand*{\updated}{\today}
\newcommand*{\firstdraft}{16 February 2019}
\newcommand*{\firstpublished}{***}
\newcommand*{\propertitle}{Posteriors for sufficiency hypotheses\\and
  maximum-entropy%\\{\large ***}%
}
\newcommand*{\pdftitle}{Posteriors for sufficiency hypotheses and maximum-entropy}
\newcommand*{\headtitle}{Posteriors for sufficiency hypotheses}
\newcommand*{\pdfauthor}{P.G.L.  Porta Mana}
\newcommand*{\headauthor}{Porta Mana}
\newcommand*{\reporthead}{}% Report number

%%%%%%%%%%%%%%%%%%%%%%%%%%%%%%%%%%%%%%%%%%%%%%%%%%%%%%%%%%%%%%%%%%%%%%%%%%%%
%%% Calls to packages (uncomment as needed)
%%%%%%%%%%%%%%%%%%%%%%%%%%%%%%%%%%%%%%%%%%%%%%%%%%%%%%%%%%%%%%%%%%%%%%%%%%%%

%\usepackage{pifont}

%\usepackage{fontawesome}

\usepackage[T1]{fontenc} 
\input{glyphtounicode} \pdfgentounicode=1

\usepackage[utf8]{inputenx}

%\usepackage{newunicodechar}
% \newunicodechar{Ĕ}{\u{E}}
% \newunicodechar{ĕ}{\u{e}}
% \newunicodechar{Ĭ}{\u{I}}
% \newunicodechar{ĭ}{\u{\i}}
% \newunicodechar{Ŏ}{\u{O}}
% \newunicodechar{ŏ}{\u{o}}
% \newunicodechar{Ŭ}{\u{U}}
% \newunicodechar{ŭ}{\u{u}}
% \newunicodechar{Ā}{\=A}
% \newunicodechar{ā}{\=a}
% \newunicodechar{Ē}{\=E}
% \newunicodechar{ē}{\=e}
% \newunicodechar{Ī}{\=I}
% \newunicodechar{ī}{\={\i}}
% \newunicodechar{Ō}{\=O}
% \newunicodechar{ō}{\=o}
% \newunicodechar{Ū}{\=U}
% \newunicodechar{ū}{\=u}
% \newunicodechar{Ȳ}{\=Y}
% \newunicodechar{ȳ}{\=y}

\newcommand*{\bmmax}{0} % reduce number of bold fonts, before font packages
\newcommand*{\hmmax}{0} % reduce number of heavy fonts, before font packages

\usepackage{textcomp}

%\usepackage[normalem]{ulem}% package for underlining
% \makeatletter
% \def\ssout{\bgroup \ULdepth=-.35ex%\UL@setULdepth
%  \markoverwith{\lower\ULdepth\hbox
%    {\kern-.03em\vbox{\hrule width.2em\kern1.2\p@\hrule}\kern-.03em}}%
%  \ULon}
% \makeatother

\usepackage{amsmath}

\usepackage{mathtools}
\addtolength{\jot}{\jot} % increase spacing in multiline formulae
\setlength{\multlinegap}{0pt}

%\usepackage{empheq}% automatically calls amsmath and mathtools
\newcommand*{\widefbox}[1]{\fbox{\hspace{1em}#1\hspace{1em}}}

%\usepackage{fancybox}

%\usepackage{framed}

% \usepackage[misc]{ifsym} % for dice
% \newcommand*{\diceone}{{\scriptsize\Cube{1}}}

\usepackage{amssymb}

\usepackage{amsxtra}

\usepackage[main=british,french,italian,german,swedish,latin,esperanto]{babel}\selectlanguage{british}
\newcommand*{\langfrench}{\foreignlanguage{french}}
\newcommand*{\langgerman}{\foreignlanguage{german}}
\newcommand*{\langitalian}{\foreignlanguage{italian}}
\newcommand*{\langswedish}{\foreignlanguage{swedish}}
\newcommand*{\langlatin}{\foreignlanguage{latin}}
\newcommand*{\langnohyph}{\foreignlanguage{nohyphenation}}

\usepackage[autostyle=false,autopunct=false,english=british]{csquotes}
\setquotestyle{british}

\usepackage{amsthm}
\newcommand*{\QED}{\textsc{q.e.d.}}
\renewcommand*{\qedsymbol}{\QED}
\theoremstyle{remark}
\newtheorem{note}{Note}
\newtheorem*{remark}{Note}
\newtheoremstyle{innote}{\parsep}{\parsep}{\footnotesize}{}{}{}{0pt}{}
\theoremstyle{innote}
\newtheorem*{innote}{}

\usepackage[shortlabels,inline]{enumitem}
\SetEnumitemKey{para}{itemindent=\parindent,leftmargin=0pt,listparindent=\parindent,parsep=0pt,itemsep=\topsep}
% \begin{asparaenum} = \begin{enumerate}[para]
% \begin{inparaenum} = \begin{enumerate*}
\setlist[enumerate,2]{label=\alph*.}
\setlist[enumerate]{label=\arabic*.,leftmargin=1.5\parindent}
\setlist[itemize]{leftmargin=1.5\parindent}
\setlist[description]{leftmargin=1.5\parindent}
% old alternative:
% \setlist[enumerate,2]{label=\alph*.}
% \setlist[enumerate]{leftmargin=\parindent}
% \setlist[itemize]{leftmargin=\parindent}
% \setlist[description]{leftmargin=\parindent}

\usepackage[babel,theoremfont,largesc]{newpxtext}

\usepackage[bigdelims,nosymbolsc%,smallerops % probably arXiv doesn't have it
]{newpxmath}
\useosf\linespread{1.083}
%% smaller operators for old version of newpxmath
\makeatletter
\def\re@DeclareMathSymbol#1#2#3#4{%
    \let#1=\undefined
    \DeclareMathSymbol{#1}{#2}{#3}{#4}}
%\re@DeclareMathSymbol{\bigsqcupop}{\mathop}{largesymbols}{"46}
%\re@DeclareMathSymbol{\bigodotop}{\mathop}{largesymbols}{"4A}
\re@DeclareMathSymbol{\bigoplusop}{\mathop}{largesymbols}{"4C}
\re@DeclareMathSymbol{\bigotimesop}{\mathop}{largesymbols}{"4E}
\re@DeclareMathSymbol{\sumop}{\mathop}{largesymbols}{"50}
\re@DeclareMathSymbol{\prodop}{\mathop}{largesymbols}{"51}
\re@DeclareMathSymbol{\bigcupop}{\mathop}{largesymbols}{"53}
\re@DeclareMathSymbol{\bigcapop}{\mathop}{largesymbols}{"54}
%\re@DeclareMathSymbol{\biguplusop}{\mathop}{largesymbols}{"55}
\re@DeclareMathSymbol{\bigwedgeop}{\mathop}{largesymbols}{"56}
\re@DeclareMathSymbol{\bigveeop}{\mathop}{largesymbols}{"57}
%\re@DeclareMathSymbol{\bigcupdotop}{\mathop}{largesymbols}{"DF}
%\re@DeclareMathSymbol{\bigcapplusop}{\mathop}{largesymbolsPXA}{"00}
%\re@DeclareMathSymbol{\bigsqcupplusop}{\mathop}{largesymbolsPXA}{"02}
%\re@DeclareMathSymbol{\bigsqcapplusop}{\mathop}{largesymbolsPXA}{"04}
%\re@DeclareMathSymbol{\bigsqcapop}{\mathop}{largesymbolsPXA}{"06}
\re@DeclareMathSymbol{\bigtimesop}{\mathop}{largesymbolsPXA}{"10}
%\re@DeclareMathSymbol{\coprodop}{\mathop}{largesymbols}{"60}
%\re@DeclareMathSymbol{\varprod}{\mathop}{largesymbolsPXA}{16}
\makeatother
%%
%% With euler font cursive for Greek letters - the [1] means 100% scaling
\DeclareFontFamily{U}{egreek}{\skewchar\font'177}%
\DeclareFontShape{U}{egreek}{m}{n}{<-6>s*[1]eurm5 <6-8>s*[1]eurm7 <8->s*[1]eurm10}{}%
\DeclareFontShape{U}{egreek}{m}{it}{<->s*[1]eurmo10}{}%
\DeclareFontShape{U}{egreek}{b}{n}{<-6>s*[1]eurb5 <6-8>s*[1]eurb7 <8->s*[1]eurb10}{}%
\DeclareFontShape{U}{egreek}{b}{it}{<->s*[1]eurbo10}{}%
\DeclareSymbolFont{egreeki}{U}{egreek}{m}{it}%
\SetSymbolFont{egreeki}{bold}{U}{egreek}{b}{it}% from the amsfonts package
\DeclareSymbolFont{egreekr}{U}{egreek}{m}{n}%
\SetSymbolFont{egreekr}{bold}{U}{egreek}{b}{n}% from the amsfonts package
% Take also \sum, \prod, \coprod symbols from Euler fonts
\DeclareFontFamily{U}{egreekx}{\skewchar\font'177}
\DeclareFontShape{U}{egreekx}{m}{n}{%
       <-7.5>s*[0.9]euex7%
    <7.5-8.5>s*[0.9]euex8%
    <8.5-9.5>s*[0.9]euex9%
    <9.5->s*[0.9]euex10%
}{}
\DeclareSymbolFont{egreekx}{U}{egreekx}{m}{n}
\DeclareMathSymbol{\sumop}{\mathop}{egreekx}{"50}
\DeclareMathSymbol{\prodop}{\mathop}{egreekx}{"51}
\DeclareMathSymbol{\coprodop}{\mathop}{egreekx}{"60}
\makeatletter
\def\sum{\DOTSI\sumop\slimits@}
\def\prod{\DOTSI\prodop\slimits@}
\def\coprod{\DOTSI\coprodop\slimits@}
\makeatother
\input{definegreek.tex}% Greek letters not usually given in LaTeX.

%\usepackage%[scaled=0.9]%
%{classico}%  Optima as sans-serif font
\usepackage{classico} % \renewcommand\sfdefault{URWClassico-TLF}
\DeclareMathAlphabet{\mathsf}  {T1}{\sfdefault}{m}{sl}
\SetMathAlphabet{\mathsf}{bold}{T1}{\sfdefault}{b}{sl}
\newcommand*{\mathte}[1]{\textbf{\textit{\textsf{#1}}}}
% Upright sans-serif math alphabet
% \DeclareMathAlphabet{\mathsu}  {T1}{\sfdefault}{m}{n}
% \SetMathAlphabet{\mathsu}{bold}{T1}{\sfdefault}{b}{n}

% DejaVu Mono as typewriter text
\usepackage[scaled=0.84]{DejaVuSansMono}

\usepackage{mathdots}

\usepackage[usenames]{xcolor}
% Tol (2012) colour-blind-, print-, screen-friendly colours, alternative scheme; Munsell terminology
\definecolor{mypurpleblue}{RGB}{68,119,170}
\definecolor{myblue}{RGB}{102,204,238}
\definecolor{mygreen}{RGB}{34,136,51}
\definecolor{myyellow}{RGB}{204,187,68}
\definecolor{myred}{RGB}{238,102,119}
\definecolor{myredpurple}{RGB}{170,51,119}
\definecolor{mygrey}{RGB}{187,187,187}
% Tol (2012) colour-blind-, print-, screen-friendly colours; Munsell terminology
% \definecolor{lbpurple}{RGB}{51,34,136}
% \definecolor{lblue}{RGB}{136,204,238}
% \definecolor{lbgreen}{RGB}{68,170,153}
% \definecolor{lgreen}{RGB}{17,119,51}
% \definecolor{lgyellow}{RGB}{153,153,51}
% \definecolor{lyellow}{RGB}{221,204,119}
% \definecolor{lred}{RGB}{204,102,119}
% \definecolor{lpred}{RGB}{136,34,85}
% \definecolor{lrpurple}{RGB}{170,68,153}
\definecolor{lgrey}{RGB}{221,221,221}
%\newcommand*\mycolourbox[1]{%
%\colorbox{mygrey}{\hspace{1em}#1\hspace{1em}}}
\colorlet{shadecolor}{lgrey}

\usepackage{bm}

\usepackage{microtype}

\usepackage[backend=biber,mcite,%subentry,
citestyle=authoryear-comp,bibstyle=pglpm-authoryear,autopunct=false,sorting=ny,sortcites=false,natbib=false,maxcitenames=1,maxbibnames=8,minbibnames=8,giveninits=true,uniquename=false,uniquelist=false,maxalphanames=1,block=space,hyperref=true,defernumbers=false,useprefix=true,sortupper=false,language=british,parentracker=false]{biblatex}
\DeclareSortingScheme{ny}{\sort{\field{sortname}\field{author}\field{editor}}\sort{\field{year}}}
\iffalse\makeatletter%%% replace parenthesis with brackets
\newrobustcmd*{\parentexttrack}[1]{%
  \begingroup
  \blx@blxinit
  \blx@setsfcodes
  \blx@bibopenparen#1\blx@bibcloseparen
  \endgroup}
\AtEveryCite{%
  \let\parentext=\parentexttrack%
  \let\bibopenparen=\bibopenbracket%
  \let\bibcloseparen=\bibclosebracket}
\makeatother\fi
\DefineBibliographyExtras{british}{\def\finalandcomma{\addcomma}}
\renewcommand*{\finalnamedelim}{\addcomma\space}
\setcounter{biburlnumpenalty}{1}
\setcounter{biburlucpenalty}{0}
\setcounter{biburllcpenalty}{1}
\DeclareDelimFormat{multicitedelim}{\addsemicolon\space}
\DeclareDelimFormat{compcitedelim}{\addsemicolon\space}
\DeclareDelimFormat{postnotedelim}{\space}
\ifarxiv\else\addbibresource{portamanabib.bib}\fi
\renewcommand{\bibfont}{\footnotesize}
%\appto{\citesetup}{\footnotesize}% smaller font for citations
\defbibheading{bibliography}[\bibname]{\section*{#1}\addcontentsline{toc}{section}{#1}%\markboth{#1}{#1}
}
\newcommand*{\citep}{\parencites}
\newcommand*{\citey}{\parencites*}
%\renewcommand*{\cite}{\parencite}
\renewcommand*{\cites}{\parencites}
\providecommand{\href}[2]{#2}
\providecommand{\eprint}[2]{\texttt{\href{#1}{#2}}}
\newcommand*{\amp}{\&}
% \newcommand*{\citein}[2][]{\textnormal{\textcite[#1]{#2}}%\addtocategory{extras}{#2}
% }
\newcommand*{\citein}[2][]{\textnormal{\textcite[#1]{#2}}%\addtocategory{extras}{#2}
}
\newcommand*{\citebi}[2][]{\textcite[#1]{#2}%\addtocategory{extras}{#2}
}
\newcommand*{\subtitleproc}[1]{}
\newcommand*{\chapb}{ch.}
%
% \def\arxivp{}
% \def\mparcp{}
% \def\philscip{}
% \def\biorxivp{}
% \newcommand*{\arxivsi}{\texttt{arXiv} eprints available at \url{http://arxiv.org/}.\\}
% \newcommand*{\mparcsi}{\texttt{mp\_arc} eprints available at \url{http://www.ma.utexas.edu/mp_arc/}.\\}
% \newcommand*{\philscisi}{\texttt{philsci} eprints available at \url{http://philsci-archive.pitt.edu/}.\\}
% \newcommand*{\biorxivsi}{\texttt{bioRxiv} eprints available at \url{http://biorxiv.org/}.\\}
\newcommand*{\arxiveprint}[1]{%\global\def\arxivp{\arxivsi}%\citeauthor{0arxivcite}\addtocategory{ifarchcit}{0arxivcite}%eprint
\texttt{\urlalt{https://arxiv.org/abs/#1}{arXiv:\hspace{0pt}#1}}%
%\texttt{\href{http://arxiv.org/abs/#1}{\protect\url{arXiv:#1}}}%
%\renewcommand{\arxivnote}{\texttt{arXiv} eprints available at \url{http://arxiv.org/}.}
}
\newcommand*{\mparceprint}[1]{%\global\def\mparcp{\mparcsi}%\citeauthor{0mparccite}\addtocategory{ifarchcit}{0mparccite}%eprint
\texttt{\urlalt{http://www.ma.utexas.edu/mp_arc-bin/mpa?yn=#1}{mp\_arc:\hspace{0pt}#1}}%
%\texttt{\href{http://www.ma.utexas.edu/mp_arc-bin/mpa?yn=#1}{\protect\url{mp_arc:#1}}}%
%\providecommand{\mparcnote}{\texttt{mp_arc} eprints available at \url{http://www.ma.utexas.edu/mp_arc/}.}
}
\newcommand*{\philscieprint}[1]{%\global\def\philscip{\philscisi}%\citeauthor{0philscicite}\addtocategory{ifarchcit}{0philscicite}%eprint
\texttt{\urlalt{http://philsci-archive.pitt.edu/archive/#1}{PhilSci:\hspace{0pt}#1}}%
%\texttt{\href{http://philsci-archive.pitt.edu/archive/#1}{\protect\url{PhilSci:#1}}}%
%\providecommand{\mparcnote}{\texttt{philsci} eprints available at \url{http://philsci-archive.pitt.edu/}.}
}
\newcommand*{\biorxiveprint}[1]{%\global\def\biorxivp{\biorxivsi}%\citeauthor{0arxivcite}\addtocategory{ifarchcit}{0arxivcite}%eprint
\texttt{\urlalt{https://doi.org/10.1101/#1}{bioRxiv doi:\hspace{0pt}10.1101/#1}}%
%\texttt{\href{http://arxiv.org/abs/#1}{\protect\url{arXiv:#1}}}%
%\renewcommand{\arxivnote}{\texttt{arXiv} eprints available at \url{http://arxiv.org/}.}
}
\newcommand*{\osfeprint}[1]{%
\texttt{\urlalt{https://doi.org/10.17605/osf.io/#1}{Open Science Framework doi:10.17605/osf.io/#1}}%
}

\usepackage{graphicx}

%\usepackage{wrapfig}

%\usepackage{tikz-cd}

\PassOptionsToPackage{hyphens}{url}\usepackage[hypertexnames=false]{hyperref}

\usepackage[depth=4]{bookmark}
\hypersetup{colorlinks=true,bookmarksnumbered,pdfborder={0 0 0.25},citebordercolor={0.2667 0.4667 0.6667},citecolor=mypurpleblue,linkbordercolor={0.6667 0.2 0.4667},linkcolor=myredpurple,urlbordercolor={0.1333 0.5333 0.2},urlcolor=mygreen,breaklinks=true,pdftitle={\pdftitle},pdfauthor={\pdfauthor}}
% \usepackage[vertfit=local]{breakurl}% only for arXiv
\providecommand*{\urlalt}{\href}

\usepackage[british]{datetime2}
\DTMnewdatestyle{mydate}%
{% definitions
\renewcommand*{\DTMdisplaydate}[4]{%
\number##3\ \DTMenglishmonthname{##2} ##1}%
\renewcommand*{\DTMDisplaydate}{\DTMdisplaydate}%
}
\DTMsetdatestyle{mydate}

%%%%%%%%%%%%%%%%%%%%%%%%%%%%%%%%%%%%%%%%%%%%%%%%%%%%%%%%%%%%%%%%%%%%%%%%%%%%
%%% Layout. I do not know on which kind of paper the reader will print the
%%% paper on (A4? letter? one-sided? double-sided?). So I choose A5, which
%%% provides a good layout for reading on screen and save paper if printed
%%% two pages per sheet. Average length line is 66 characters and page
%%% numbers are centred.
%%%%%%%%%%%%%%%%%%%%%%%%%%%%%%%%%%%%%%%%%%%%%%%%%%%%%%%%%%%%%%%%%%%%%%%%%%%%
\ifafour\setstocksize{297mm}{210mm}%{*}% A4
\else\setstocksize{210mm}{5.5in}%{*}% 210x139.7
\fi
\settrimmedsize{\stockheight}{\stockwidth}{*}
\setlxvchars[\normalfont] %313.3632pt for a 66-characters line
\setxlvchars[\normalfont]
\setlength{\trimtop}{0pt}
\setlength{\trimedge}{\stockwidth}
\addtolength{\trimedge}{-\paperwidth}
% The length of the normalsize alphabet is 133.05988pt - 10 pt = 26.1408pc
% The length of the normalsize alphabet is 159.6719pt - 12pt = 30.3586pc
% Bringhurst gives 32pc as boundary optimal with 69 ch per line
% The length of the normalsize alphabet is 191.60612pt - 14pt = 35.8634pc
\ifafour\settypeblocksize{*}{32pc}{1.618} % A4
%\setulmargins{*}{*}{1.667}%gives 5/3 margins % 2 or 1.667
\else\settypeblocksize{*}{26pc}{1.618}% nearer to a 66-line newpx and preserves GR
\fi
\setulmargins{*}{*}{1}%gives equal margins
\setlrmargins{*}{*}{*}
\setheadfoot{\onelineskip}{2.5\onelineskip}
\setheaderspaces{*}{2\onelineskip}{*}
\setmarginnotes{2ex}{10mm}{0pt}
\checkandfixthelayout[nearest]
\fixpdflayout
%%% End layout
%% this fixes missing white spaces
\pdfmapline{+dummy-space <dummy-space.pfb}\pdfinterwordspaceon%

%%% Sectioning
\newcommand*{\asudedication}[1]{%
{\par\centering\textit{#1}\par}}
\newenvironment{acknowledgements}{\section*{Thanks}\addcontentsline{toc}{section}{Thanks}}{\par}
\makeatletter\renewcommand{\appendix}{\par
  \bigskip{\centering
   \interlinepenalty \@M
   \normalfont
   \printchaptertitle{\sffamily\appendixpagename}\par}
  \setcounter{section}{0}%
  \gdef\@chapapp{\appendixname}%
  \gdef\thesection{\@Alph\c@section}%
  \anappendixtrue}\makeatother
\counterwithout{section}{chapter}
\setsecnumformat{\upshape\csname the#1\endcsname\quad}
\setsecheadstyle{\large\bfseries\sffamily%
\centering}
\setsubsecheadstyle{\bfseries\sffamily%
\raggedright}
%\setbeforesecskip{-1.5ex plus 1ex minus .2ex}% plus 1ex minus .2ex}
%\setaftersecskip{1.3ex plus .2ex }% plus 1ex minus .2ex}
%\setsubsubsecheadstyle{\bfseries\sffamily\slshape\raggedright}
%\setbeforesubsecskip{1.25ex plus 1ex minus .2ex }% plus 1ex minus .2ex}
%\setaftersubsecskip{-1em}%{-0.5ex plus .2ex}% plus 1ex minus .2ex}
\setsubsecindent{0pt}%0ex plus 1ex minus .2ex}
\setparaheadstyle{\bfseries\sffamily%
\raggedright}
\setcounter{secnumdepth}{2}
\setlength{\headwidth}{\textwidth}
\newcommand{\addchap}[1]{\chapter*[#1]{#1}\addcontentsline{toc}{chapter}{#1}}
\newcommand{\addsec}[1]{\section*{#1}\addcontentsline{toc}{section}{#1}}
\newcommand{\addsubsec}[1]{\subsection*{#1}\addcontentsline{toc}{subsection}{#1}}
\newcommand{\addpara}[1]{\paragraph*{#1.}\addcontentsline{toc}{subsubsection}{#1}}
\newcommand{\addparap}[1]{\paragraph*{#1}\addcontentsline{toc}{subsubsection}{#1}}

%%% Headers, footers, pagestyle
\copypagestyle{manaart}{plain}
\makeheadrule{manaart}{\headwidth}{0.5\normalrulethickness}
\makeoddhead{manaart}{%
{\footnotesize%\sffamily%
\scshape\headauthor}}{}{{\footnotesize\sffamily%
\headtitle}}
\makeoddfoot{manaart}{}{\thepage}{}
\newcommand*\autanet{\includegraphics[height=\heightof{M}]{autanet.pdf}}
\definecolor{mygray}{gray}{0.333}
\iftypodisclaim%
\ifafour\newcommand\addprintnote{\begin{picture}(0,0)%
\put(245,149){\makebox(0,0){\rotatebox{90}{\tiny\color{mygray}\textsf{This
            document is designed for screen reading and
            two-up printing on A4 or Letter paper}}}}%
\end{picture}}% A4
\else\newcommand\addprintnote{\begin{picture}(0,0)%
\put(176,112){\makebox(0,0){\rotatebox{90}{\tiny\color{mygray}\textsf{This
            document is designed for screen reading and
            two-up printing on A4 or Letter paper}}}}%
\end{picture}}\fi%afourtrue
\makeoddfoot{plain}{}{\makebox[0pt]{\thepage}\addprintnote}{}
\else
\makeoddfoot{plain}{}{\makebox[0pt]{\thepage}}{}
\fi%typodisclaimtrue
\makeoddhead{plain}{}{}{\footnotesize\reporthead}
% \copypagestyle{manainitial}{plain}
% \makeheadrule{manainitial}{\headwidth}{0.5\normalrulethickness}
% \makeoddhead{manainitial}{%
% \footnotesize\sffamily%
% \scshape\headauthor}{}{\footnotesize\sffamily%
% \headtitle}
% \makeoddfoot{manaart}{}{\thepage}{}

\pagestyle{manaart}

\setlength{\droptitle}{-3.9\onelineskip}
\pretitle{\begin{center}\Large\sffamily%
\bfseries}
\posttitle{\bigskip\end{center}}

\makeatletter\newcommand*{\atf}{\includegraphics[%trim=1pt 1pt 0pt 0pt,
totalheight=\heightof{@}]{atblack.png}}\makeatother
\providecommand{\affiliation}[1]{\textsl{\textsf{\footnotesize #1}}}
\providecommand{\epost}[1]{\texttt{\footnotesize\textless#1\textgreater}}
\providecommand{\email}[2]{\href{mailto:#1ZZ@#2 ((remove ZZ))}{#1\protect\atf#2}}

\preauthor{\vspace{-0.5\baselineskip}\begin{center}
\normalsize\sffamily%
\lineskip  0.5em}
\postauthor{\par\end{center}}
\predate{\DTMsetdatestyle{mydate}\begin{center}\footnotesize}
\postdate{\end{center}\vspace{-\medskipamount}}

\setfloatadjustment{figure}{\footnotesize}
\captiondelim{\quad}
\captionnamefont{\footnotesize\sffamily%
}
\captiontitlefont{\footnotesize}
\firmlists*
\midsloppy
% handling orphan/widow lines, memman.pdf
% \clubpenalty=10000
% \widowpenalty=10000
% \raggedbottom
% Downes, memman.pdf
\clubpenalty=9996
\widowpenalty=9999
\brokenpenalty=4991
\predisplaypenalty=10000
\postdisplaypenalty=1549
\displaywidowpenalty=1602
\selectlanguage{british}\frenchspacing

%%%%%%%%%%%%%%%%%%%%%%%%%%%%%%%%%%%%%%%%%%%%%%%%%%%%%%%%%%%%%%%%%%%%%%%%%%%%
%%% Paper's details
%%%%%%%%%%%%%%%%%%%%%%%%%%%%%%%%%%%%%%%%%%%%%%%%%%%%%%%%%%%%%%%%%%%%%%%%%%%%
\title{\propertitle}
\author{%
\hspace*{\stretch{1}}%
%% uncomment if additional authors present
% \parbox{0.5\linewidth}%\makebox[0pt][c]%
% {\protect\centering ***\\%
% \footnotesize\epost{\email{***}{***}}}%
% \hspace*{\stretch{1}}%
\parbox{0.5\linewidth}%\makebox[0pt][c]%
{\protect\centering P.G.L.  Porta Mana\\%
\footnotesize\epost{\email{piero.mana}{ntnu.no}}}%
\hspace*{\stretch{1}}%
%\quad\href{https://orcid.org/0000-0002-6070-0784}{\protect\includegraphics[scale=0.16]{orcid_32x32.png}\textsc{orcid}:0000-0002-6070-0784}%
}

\date{Draft of \today\ (first drafted \firstdraft)}
%\date{\firstpublished; updated \updated}

%%%%%%%%%%%%%%%%%%%%%%%%%%%%%%%%%%%%%%%%%%%%%%%%%%%%%%%%%%%%%%%%%%%%%%%%%%%%
%%% Macros @@@
%%%%%%%%%%%%%%%%%%%%%%%%%%%%%%%%%%%%%%%%%%%%%%%%%%%%%%%%%%%%%%%%%%%%%%%%%%%%

% Common ones - uncomment as needed
%\providecommand{\nequiv}{\not\equiv}
%\providecommand{\coloneqq}{\mathrel{\mathop:}=}
%\providecommand{\eqqcolon}{=\mathrel{\mathop:}}
%\providecommand{\varprod}{\prod}
\newcommand*{\de}{\partialup}%partial diff
\newcommand*{\pu}{\piup}%constant pi
\newcommand*{\delt}{\deltaup}%Kronecker, Dirac
%\newcommand*{\eps}{\varepsilonup}%Levi-Civita, Heaviside
%\newcommand*{\riem}{\zetaup}%Riemann zeta
%\providecommand{\degree}{\textdegree}% degree
%\newcommand*{\celsius}{\textcelsius}% degree Celsius
%\newcommand*{\micro}{\textmu}% degree Celsius
\newcommand*{\I}{\mathrm{i}}%imaginary unit
\newcommand*{\e}{\mathrm{e}}%Neper
\newcommand*{\di}{\mathrm{d}}%differential
%\newcommand*{\Di}{\mathrm{D}}%capital differential
%\newcommand*{\planckc}{\hslash}
%\newcommand*{\avogn}{N_{\textrm{A}}}
%\newcommand*{\NN}{\bm{\mathrm{N}}}
%\newcommand*{\ZZ}{\bm{\mathrm{Z}}}
%\newcommand*{\QQ}{\bm{\mathrm{Q}}}
\newcommand*{\RR}{\bm{\mathrm{R}}}
%\newcommand*{\CC}{\bm{\mathrm{C}}}
%\newcommand*{\nabl}{\bm{\nabla}}%nabla
%\DeclareMathOperator{\lb}{lb}%base 2 log
%\DeclareMathOperator{\tr}{tr}%trace
%\DeclareMathOperator{\card}{card}%cardinality
%\DeclareMathOperator{\im}{Im}%im part
%\DeclareMathOperator{\re}{Re}%re part
%\DeclareMathOperator{\sgn}{sgn}%signum
%\DeclareMathOperator{\ent}{ent}%integer less or equal to
%\DeclareMathOperator{\Ord}{O}%same order as
%\DeclareMathOperator{\ord}{o}%lower order than
%\newcommand*{\incr}{\triangle}%finite increment
\newcommand*{\defd}{\coloneqq}
\newcommand*{\defs}{\eqqcolon}
%\newcommand*{\Land}{\bigwedge}
%\newcommand*{\Lor}{\bigvee}
%\newcommand*{\lland}{\DOTSB\;\land\;}
%\newcommand*{\llor}{\DOTSB\;\lor\;}
%\newcommand*{\limplies}{\mathbin{\Rightarrow}}%implies
%\newcommand*{\suchthat}{\mid}%{\mathpunct{|}}%such that (eg in sets)
%\newcommand*{\with}{\colon}%with (list of indices)
%\newcommand*{\mul}{\times}%multiplication
%\newcommand*{\inn}{\cdot}%inner product
\newcommand*{\dotv}{\mathord{\,\cdot\,}}%variable place
%\newcommand*{\comp}{\circ}%composition of functions
%\newcommand*{\con}{\mathbin{:}}%scal prod of tensors
%\newcommand*{\equi}{\sim}%equivalent to 
\renewcommand*{\asymp}{\simeq}%equivalent to 
%\newcommand*{\corr}{\mathrel{\hat{=}}}%corresponds to
%\providecommand{\varparallel}{\ensuremath{\mathbin{/\mkern-7mu/}}}%parallel (tentative symbol)
\renewcommand*{\le}{\leqslant}%less or equal
\renewcommand*{\ge}{\geqslant}%greater or equal
%\DeclarePairedDelimiter\clcl{[}{]}
%\DeclarePairedDelimiter\clop{[}{[}
%\DeclarePairedDelimiter\opcl{]}{]}
%\DeclarePairedDelimiter\opop{]}{[}
\DeclarePairedDelimiter\abs{\lvert}{\rvert}
%\DeclarePairedDelimiter\norm{\lVert}{\rVert}
\DeclarePairedDelimiter\set{\{}{\}}
%\DeclareMathOperator{\pr}{P}%probability
\newcommand*{\pf}{\mathrm{p}}%probability
\newcommand*{\p}{\mathrm{P}}%probability
\newcommand*{\E}{\mathrm{E}}
\renewcommand*{\|}{\nonscript\,\vert\nonscript\;\mathopen{}}
\DeclarePairedDelimiterX{\cond}[2]{(}{)}{#1\nonscript\,\delimsize\vert\nonscript\;\mathopen{}#2}
%\DeclarePairedDelimiterX{\condt}[2]{[}{]}{#1\nonscript\,\delimsize\vert\nonscript\;\mathopen{}#2}
%\DeclarePairedDelimiterX{\conds}[2]{\{}{\}}{#1\nonscript\,\delimsize\vert\nonscript\;\mathopen{}#2}
%\newcommand*{\+}{\lor}
%\renewcommand{\*}{\land}
\newcommand*{\sect}{\S}% Sect.~
\newcommand*{\sects}{\S\S}% Sect.~
\newcommand*{\chap}{ch.}%
\newcommand*{\chaps}{chs}%
\newcommand*{\bref}{ref.}%
\newcommand*{\brefs}{refs}%
%\newcommand*{\fn}{fn}%
\newcommand*{\eqn}{eq.}%
\newcommand*{\eqns}{eqs}%
\newcommand*{\fig}{fig.}%
\newcommand*{\figs}{figs}%
\newcommand*{\vs}{{vs}}
%\newcommand*{\etc}{{etc.}}
%\newcommand*{\ie}{{i.e.}}
%\newcommand*{\ca}{{c.}}
%\newcommand*{\eg}{{e.g.}}
\newcommand*{\foll}{{ff.}}
%\newcommand*{\viz}{{viz}}
\newcommand*{\cf}{{cf.}}
%\newcommand*{\Cf}{{Cf.}}
%\newcommand*{\vd}{{v.}}
\newcommand*{\etal}{{et al.}}
%\newcommand*{\etsim}{{et sim.}}
%\newcommand*{\ibid}{{ibid.}}
%\newcommand*{\sic}{{sic}}
%\newcommand*{\id}{\mathte{I}}%id matrix
%\newcommand*{\nbd}{\nobreakdash}%
%\newcommand*{\bd}{\hspace{0pt}}%
%\def\hy{-\penalty0\hskip0pt\relax}
%\newcommand*{\labelbis}[1]{\tag*{(\ref{#1})$_\text{r}$}}
%\newcommand*{\mathbox}[2][.8]{\parbox[t]{#1\columnwidth}{#2}}
%\newcommand*{\zerob}[1]{\makebox[0pt][l]{#1}}
\newcommand*{\tprod}{\mathop{\textstyle\prod}\nolimits}
\newcommand*{\tsum}{\mathop{\textstyle\sum}\nolimits}
%\newcommand*{\tint}{\begingroup\textstyle\int\endgroup\nolimits}
%\newcommand*{\tland}{\mathop{\textstyle\bigwedge}\nolimits}
%\newcommand*{\tlor}{\mathop{\textstyle\bigvee}\nolimits}
%\newcommand*{\sprod}{\mathop{\textstyle\prod}}
%\newcommand*{\ssum}{\mathop{\textstyle\sum}}
%\newcommand*{\sint}{\begingroup\textstyle\int\endgroup}
%\newcommand*{\sland}{\mathop{\textstyle\bigwedge}}
%\newcommand*{\slor}{\mathop{\textstyle\bigvee}}
%\newcommand*{\T}{^\intercal}%transpose
%%\newcommand*{\QEM}%{\textnormal{$\Box$}}%{\ding{167}}
%\newcommand*{\qem}{\leavevmode\unskip\penalty9999 \hbox{}\nobreak\hfill
%\quad\hbox{\QEM}}

%%%%%%%%%%%%%%%%%%%%%%%%%%%%%%%%%%%%%%%%%%%%%%%%%%%%%%%%%%%%%%%%%%%%%%%%%%%%
%%% Custom macros for this file @@@
%%%%%%%%%%%%%%%%%%%%%%%%%%%%%%%%%%%%%%%%%%%%%%%%%%%%%%%%%%%%%%%%%%%%%%%%%%%%
 \definecolor{notecolour}{RGB}{68,170,153}
\newcommand*{\puzzle}{{\fontencoding{U}\fontfamily{fontawesometwo}\selectfont\symbol{225}}}
%\newcommand*{\puzzle}{\maltese}
\newcommand{\mynote}[1]{ {\color{notecolour}\puzzle\ #1}}
\newcommand*{\widebar}[1]{{\mkern1.5mu\skew{2}\overline{\mkern-1.5mu#1\mkern-1.5mu}\mkern 1.5mu}}

% \newcommand{\explanation}[4][t]{%\setlength{\tabcolsep}{-1ex}
% %\smash{
% \begin{tabular}[#1]{c}#2\\[0.5\jot]\rule{1pt}{#3}\\#4\end{tabular}}%}
\newcommand*{\ptext}[1]{\text{\small #1}}
\DeclareMathOperator*{\arginf}{arg\,inf}
\newcommand*{\dob}{degree of belief}
\newcommand*{\dobs}{degrees of belief}
\newcommand*{\yS}{S}
\newcommand*{\ySt}{\bm{\yS}}
\newcommand*{\ys}{s}
\newcommand*{\yst}{\bm{\ys}}
\newcommand*{\yll}{\lambda}
\newcommand*{\yl}{\bm{\lambda}}
\newcommand*{\yC}{\mathte{C}}
\newcommand*{\ycc}{\widebar{C}}
\newcommand*{\yc}{\widebar{\bm{C}}}
\newcommand*{\yG}{G}
\newcommand*{\yg}{g}
\newcommand*{\ygt}{\widebar{\yg}}
\newcommand*{\yI}{\varIota}
\newcommand*{\ya}{\ycc}
\newcommand*{\yF}{\bm{F}}
\newcommand*{\yf}{\bm{f}}
\newcommand*{\yH}{\varEta}
\newcommand*{\yCs}{\mathte{c}}
\newcommand*{\yccs}{\widebar{c}}
\newcommand*{\ycs}{\widebar{\bm{c}}}
%%% Custom macros end @@@

%%%%%%%%%%%%%%%%%%%%%%%%%%%%%%%%%%%%%%%%%%%%%%%%%%%%%%%%%%%%%%%%%%%%%%%%%%%%
%%% Beginning of document
%%%%%%%%%%%%%%%%%%%%%%%%%%%%%%%%%%%%%%%%%%%%%%%%%%%%%%%%%%%%%%%%%%%%%%%%%%%%
\firmlists
\begin{document}
\captiondelim{\quad}\captionnamefont{\footnotesize}\captiontitlefont{\footnotesize}
\selectlanguage{british}\frenchspacing
\maketitle

%%%%%%%%%%%%%%%%%%%%%%%%%%%%%%%%%%%%%%%%%%%%%%%%%%%%%%%%%%%%%%%%%%%%%%%%%%%%
%%% Abstract
%%%%%%%%%%%%%%%%%%%%%%%%%%%%%%%%%%%%%%%%%%%%%%%%%%%%%%%%%%%%%%%%%%%%%%%%%%%%
\abstractrunin
\abslabeldelim{}
\renewcommand*{\abstractname}{}
\setlength{\absleftindent}{0pt}
\setlength{\absrightindent}{0pt}
\setlength{\abstitleskip}{-\absparindent}
\begin{abstract}\labelsep 0pt%
  \noindent Assessing the probability of a hypothesis of sufficiency from
  the observation of a sample.
\iffalse\\\noindent\emph{\footnotesize Note: Dear Reader
    \amp\ Peer, this manuscript is being peer-reviewed by you. Thank you.}\fi
% \par%\\[\jot]
% \noindent
% {\footnotesize PACS: ***}\qquad%
% {\footnotesize MSC: ***}%
%\qquad{\footnotesize Keywords: ***}
\end{abstract}
\selectlanguage{british}\frenchspacing

%%%%%%%%%%%%%%%%%%%%%%%%%%%%%%%%%%%%%%%%%%%%%%%%%%%%%%%%%%%%%%%%%%%%%%%%%%%%
%%% Epigraph
%%%%%%%%%%%%%%%%%%%%%%%%%%%%%%%%%%%%%%%%%%%%%%%%%%%%%%%%%%%%%%%%%%%%%%%%%%%%
% \asudedication{\small ***}
% \vspace{\bigskipamount}
% \setlength{\epigraphwidth}{.7\columnwidth}
% %\epigraphposition{flushright}
% \epigraphtextposition{flushright}
% %\epigraphsourceposition{flushright}
% \epigraphfontsize{\footnotesize}
% \setlength{\epigraphrule}{0pt}
% %\setlength{\beforeepigraphskip}{0pt}
% %\setlength{\afterepigraphskip}{0pt}
% \epigraph{\emph{text}}{source}



%%%%%%%%%%%%%%%%%%%%%%%%%%%%%%%%%%%%%%%%%%%%%%%%%%%%%%%%%%%%%%%%%%%%%%%%%%%%
%%% BEGINNING OF MAIN TEXT
%%%%%%%%%%%%%%%%%%%%%%%%%%%%%%%%%%%%%%%%%%%%%%%%%%%%%%%%%%%%%%%%%%%%%%%%%%%%

\section{Hypotheses about sufficient statistics}
\label{sec:hypotheses}



We have a population of $N$ neurons whose activities we imagine to have
time-binned into $T$ bins and binarized. Denote their total population
activity at time bin $t$ by $\yS_{t} \in \set{0,1,\dotsc,N}$, their total
activity at an unspecified bin by $\yS$, and the time sequence of total
activities by $\ySt\defd ( \yS_{t_1}, \yS_{t_2}, \dotsc, \yS_{t_T} )$.

We have recorded the activities of a sample of $n$ neurons from the
population above. Denote the total activity of the sample at time bin $t$
by $\ys_{t} \in \set{0,1,\dotsc,n}$, at an unspecified bin by $\ys$, and the
time sequence by $\yst \defd ( \ys_{t_1}, \ys_{t_2}, \dotsc, \ys_{t_T} )$.


We don't know how these sampled neurons were chosen from the full
population. This fact leads, for each time bin, to the following \dob\
about the activity of the sample if we knew the activity of the
full population \citep[\sect~2.3]{portamanaetal2015}[\sect~2]{portamanaetal2018b}:
\begin{equation}
  \label{eq:dob_sample_given_pop}
  \pf(\ys \| \yS, \yI) = \binom{n}{\ys} \binom{N-n}{\yS-\ys} \binom{N}{\yS}^{-1}
  \defs G_{\ys\yS},
\end{equation}
namely, a hypergeometryc distribution.

Here and in the following $\yI$ denotes the proposition stating our
background information.

\bigskip

Now suppose that we knew the total activities $\ySt$ of the \emph{full}
population at some $T$ time bins $\set{t}$, and we wanted to infer the
total activities $\ySt'$ at $T'$ \emph{different} time bins $\set{t'}$:
\begin{equation}
  \label{eq:goal_new_bins_from_known}
  \pf(\ySt' \| \ySt, \yI).
\end{equation}

We want to consider the hypotheses that \emph{only a specific set of
  statistics} about our data $\ySt$ are \emph{relevant} for our inference
about $\ySt'$; that is, they are a sufficient statistics. Any aspect of the
data not contained in those statistics would be irrelevant for our
inference. This inferential property could be the result of biological
properties of the population.

Let's assume that there are $R$ such statistics (besides $T$, which is
always part of a set of sufficient statistics). Each statistic is the sum
over time of a specific function of the total activity $\yS$. We can
arrange these functions in an $R$-by-$(N+1)$ matrix $\yC \defd (C_{r\yS})$,
where $C_{r\yS}$ is the value of the function for the $r$th statistic when
the total activity is $\yS$. The $R$ sufficient statistics for the data
$\ySt$ would thus be
\begin{equation}
  \label{eq:suff_statistics_t}
\ycc_{r} \defd  \frac{1}{T}\sum_{t} C_{r\yS_{t}},\quad r \in \set{1, \dotsc, R}.
\end{equation}

\bigskip

Our goal is to quantify our uncertainty about these hypotheses of
sufficient statistics, given the activity data from a sample of neurons.
It's important to note that the hypotheses we must consider are not
discrete or of a yes-or-no type: they form a continuum. This is because we
have a continuum of degrees of relevance. Consider for example two
statistics $\ya_1$ and $\ya_2$ from the bins $\set{t}$. Our \dobs\ about
the activities at bins $\set{t'}$ are
\begin{equation}
  \label{eq:continuum_stat_example}
  \pf(\ySt' \| \ya_1, \ya_2, \yI).
\end{equation}
It may happen that lack of knowledge about $\ya_2$ doesn't change our \dob:
\begin{equation}
  \label{eq:continuum_stat_example_irrel}
  \pf(\ySt' \| \ya_1, \yI) =
  \pf(\ySt' \| \ya_1, \ya_2, \yI),
\end{equation}
in which case $\ya_2$ is irrelevant. It may also happen that our \dob\ is
changed but in a negligible way, for all values of $\ySt'$ and $\ya_1$:
\begin{equation}
  \label{eq:continuum_stat_example_irrel_slightly}
  \pf(\ySt' \| \ya_1, \yI) \approx
  \pf(\ySt' \| \ya_1, \ya_2, \yI),
\end{equation}
so that $\ya_2$ could be dropped in practice. We can imagine larger and
larger changes to the point where dropping $\ya_2$ would lead to
drastically different \dobs. The question of the relevance of $\ya_2$ is
therefore not dichotomous. We will thus deal with a continuum of
hypotheses, each representing a degree of relevance of some statistics. We
shall shortly see how to mathematically represent this continuum of
hypotheses.

\section{The Koopman-Pitman theorem}
\label{sec:koopman-pitman}


How does a hypothesis about a sufficient statistic affect our \dobs? The
answer comes from the Koopman-Pitman theorem
\citep{koopman1936,pitman1936}[see
also][]{darmois1935,barankinetal1963,denny1967,hipp1974,lauritzen1974,lauritzen1984,lauritzen1982_r1988}[for
the discrete version:][]{fraser1963,andersen1970}, which says that the
\dob~\eqref{eq:goal_new_bins_from_known} has a very specific mathematical
expression if only some statistics of $\ySt$ relevant. The main statement
of the theorem is this: if $R$ sufficient statistics are given by functions
$C_{r\yS}$, then for any number of time bins $T$
\begin{subequations}
  \label{eq:koopman-pitman}
  \begin{gather}
    \pf( \ySt \| \yI) =
    \int\!\di\yl\; \pf(\yl \| \yI)\; \prod_{t}\pf(\yS_{t} \| \yl, \yI)
    \label{eq:koopman-pitman_integral}
    \\
    \shortintertext{with}
    \pf(\yS \| \yl, \yI) \defd \frac{\yg_{\yS}}{Z(\yl)}
    \exp\bigl(\tsum_{r} \yll_r C_{r\yS}\bigr),
    \label{eq:koopman-pitman_likelihood}
    \\
    \yl \defd (\yll_1, \dotsc, \yll_R) \in \RR^{R},
    \qquad
    Z(\yl) \defd \sum_{\yS}
    \yg_{\yS}\;\exp\bigl(\tsum_{r} \yll_r C_{r\yS}\bigr),
  \end{gather}
\end{subequations}
and $\yg$ a positive function of $\yS$.

Some important remarks about the Pitman-Koopman
formula~\eqref{eq:koopman-pitman}:
\begin{enumerate}[label=\alph*.]
\item A hypothesis that only stated what the sufficient statistics are
  would not determine the density $\pf(\yl \| \yI)$ or the function $\yg$
  in the formula above. The hypotheses we are going to compare thus contain
  additional information besides sufficiency.
\item The formula can be interpreted this way: our \dob\ about $\ySt$ is
  given by the \dob\ we would have if we knew the values of the sufficient
  statistics $\yc$ for an unlimited number of time bins, mixed over our
  uncertainty about the values themselves:
  \begin{equation}
    \label{eq:koopman-pitman_interpreted}
        \pf( \ySt \| \yI) =
        \int\!\di\yc\; \pf(\yc \| \yI)\;
        \pf(\ySt \| \yc, \yI).
  \end{equation}
\item \label{item:reparametrization}Formula~\eqref{eq:koopman-pitman} is
  obtained from~\eqref{eq:koopman-pitman_interpreted} by a one-to-one
  reparametrization:
\begin{gather}
  \label{eq:pitman-koopman_reparametrization}
  \ycc_{r}(\yl) = \sum_{\yS} C_{r\yS} \frac{\yg_{\yS}}{Z(\yl)}
  \exp\bigl(\tsum_{r} \yll_{r} C_{r\yS}\bigr)
  \equiv \de_{\yll_{r}} \ln Z(\yl),
  \\
  \pf(\yc \| \yI)\,\di\yc = \pf(\yl \| \yI)\,\di\yl,
  \\
  \pf[\yS \| \yc(\yl), \yI] = \pf(\yS \| \yl, \yI) \equiv
  \frac{\yg_{\yS}}{Z(\yl)}\;\exp\bigl(\tsum_{r} \yll_r C_{r\yS}\bigr).
\end{gather}
Equation~\eqref{eq:pitman-koopman_reparametrization} cannot be solved
explicitly for $\yl$ in terms of $\yc$ except for very simple cases. The
parametrization in terms of $\yl$ has several special properties:
\begin{enumerate}[label=\alph{enumi}.\arabic*.]
\item The quantities $(\yll_{r})$ can assume any values independently of
  one another, whereas the limit statistics $(\ycc_r)$ have interdependent
  ranges.
\item The expression for $\pf(\ySt \| \yc, \yI)$ can be written explicitly
  in terms of $\yl$, but not in terms of $\yc$.
\item If some $\yll_{r}$ vanishes then the corresponding statistic is
  \emph{irrelevant} -- the corresponding term indeed disappears from the
  exponential in formula~\eqref{eq:koopman-pitman}.
\end{enumerate}
\end{enumerate}


Remark \ref{item:reparametrization} suggests that the absolute value of
each parameter, $\abs{\yll_{r}}$, might be used to quantify the degree of
relevance of the corresponding statistic, zero meaning complete
irrelevance. These absolute values can therefore be used as parameters for
our continuum of hypotheses about the sufficient statistics. The \dob\ we
ultimately want to quantify is thus
\begin{equation}
  \label{eq:posterior_hypotheses_data}
  \pf\cond[\big]{\abs{\yl}}{\ptext{data}, \yI}
\end{equation}
and its marginals for the various $\abs{\yll_{r}}$.

\section{Conditionalization on full-population data}
\label{sec:full-pop_data}

First let us recapitulate the expression for
\begin{equation}
  \label{eq:posterior_given_full_population}
  \pf(\yl \| \ySt, \yI) \propto \pf(\yl \| \yI)\;
  \pf(\ySt \| \yl, \yI).
\end{equation}
Introduce the relative frequencies $\yF \defd (F_{\yS})$ with which the
activity $\yS$ appears during the $T$ time bins, given by
\begin{equation}
  \label{eq:frequencies}
  F_{\yS} = \frac{1}{T}\sum_{t}\delt(\yS_{t}-\yS).
\end{equation}
From formula~\eqref{eq:koopman-pitman}, bringing several terms within the
exponential, we can rewrite, using the frequencies,
\begin{equation}
  \label{eq:to_maxent}
  \begin{split}
  \pf(\ySt \| \yl, \yI) &= \prod_{t} \pf(\yS_{t} \| \yl, \yI)
    % \prod_{t} \frac{\yg_{\yS_{t}}}{Z(\yl)}
    % \exp\bigl(\tsum_{r} \yll_r C_{r\yS_{t}}\bigr),
    \\ &=
    % \prod_{\yS} \Biggl[ \frac{\yg_{\yS}}{Z(\yl)}
    % \exp\bigl(\tsum_{r} \yll_r C_{r\yS}\bigr) \Biggr]^{T F_{\yS}},
    \prod_{\yS}\bigl[ \pf(\yS \| \yl, \yI) \bigr]^{T F_{\yS}}
    % \\ &=
    % \exp\{-\tsum_t[
    % \ln Z(\yl) - \ln \yg_{\yS_{t}} -\tsum_{r} \yll_r C_{r\yS_{t}}]\}
    % \\ &=
    % \exp\{-\tsum_{\yS}F_{\yS}\,[
    % \ln Z(\yl) - \ln \yg_{\yS_{t}} -\tsum_{r} \yll_r C_{r\yS_{t}}]\}
    \\ &=
    \exp\Bigl(-T\bigl\{
    \yH\bigl[\yF, \pf(\yS \| \yl, \yI)\bigr]
    -\yH(\yF)\bigr\}\Bigr),
    % \exp\biggl\{-T\,
    % \yH\biggl[\yF,
    % \frac{\yg_{\yS}}{Z(\yl)}\exp\bigl(\tsum_{r} \yll_r C_{r\yS}\bigr)\biggr]
    % -T\yH(\yF)\biggr\},
\end{split}
\end{equation}
where $\yH(\bm{a},\bm{b})$ is the relative entropy of $\bm{a}$ with respect
to $\bm{b}$ and $\yH(\bm{a})$ is the Shannon entropy of $\bm{a}$.

The last expression shows that:
\begin{enumerate*}[label=(\alph*),mode=unboxed]
\item our \dob\ about the sequence $\ySt$ depends only on the frequencies
  $\yF$ of that sequence;
\item if $\pf(\yl \| \yI)$ is constant, the mode of the
  density~\eqref{eq:posterior_given_full_population} for $\yl$ can be
  calculated by minimizing with respect to $\yl$ the relative entropy
  between $\yF$ and $\pf(\yS \| \yl, \yI)$;
\item owing to the mathematical form~\eqref{eq:koopman-pitman_likelihood}
  of $\pf(\yS \| \yl, \yI)$, minimization of the relative entropy leads to
  the classical maximum-entropy equations
  \begin{equation}
    \label{eq:standard_maxent_eqns}
    \E(C_{r\yS} \| \yl, \yI) \equiv
        \de_{\yll_{r}}\ln Z(\yl) = \yc_{r} \equiv \tsum_{\yS} C_{r\yS} F_{\yS}.
  \end{equation}
\end{enumerate*}

\section{Conditionalization on sample data}
\label{sec:sample_data}


The data for which we want to calculate the
\dob~\eqref{eq:posterior_hypotheses_data} do not involve the full
population, though, but only a sample of it: they are a sequence $\yst$ of
recorded sample activities:
\begin{equation}
  \label{eq:posterior_given_sample}
  \pf(\yl \| \yst, \yI) \propto \pf(\yl \| \yI)\;
  \pf(\yst \| \yl, \yI).
\end{equation}

Let's start by checking how our \dob\ about a sequence $\yst$ of sample
activities looks like under the hypothesis of sufficiency \emph{for the
  full population}. We can calculate it by marginalizing with respect to
all possible \emph{sequences} of activities $\ySt$ for the full population, using
the hypergeometric distribution~\eqref{eq:dob_sample_given_pop} and the
Koopman-Pitman formula~\eqref{eq:koopman-pitman}:
\begin{subequations}
  \label{eq:koopman-pitman_sample}
  \begin{gather}
      \pf( \yst \| \yI) =\sum_{\ySt} \pf(\yst \| \ySt, \yI)\;\pf(\ySt \| \yI)
=    \int\!\di\yl\; \pf(\yl \| \yI)\; \prod_{t}\pf(\ys_{t} \| \yl, \yI)
    \label{eq:koopman-pitman_integral_sample}
    \\
    \shortintertext{with}
    \pf(\ys \| \yl, \yI) \defd
\sum_{\yS} G_{\ys\yS}\;
    \frac{\yg_{\yS}}{Z(\yl)}
    \exp\bigl(\tsum_{r} \yll_r C_{r\yS}\bigr).
    \label{eq:koopman-pitman_likelihood_sample}
    % \\
    % \yl \defd (\yll_1, \dotsc, \yll_R) \in \RR^{R},
    % \qquad
    % Z(\yl) \defd \sum_{\yS}
    % \yg_{\yS}\;\exp\bigl(\tsum_{r} \yll_r C_{r\yS}\bigr),
  \end{gather}
\end{subequations}
This formula shows that our \dob\ about the sample activities does
\emph{not} have a sufficient statistics. This fact is mathematically
similar to what happens in statistical mechanics: if our uncertainty about
a system of particles is expressed by a Gibbs distribution, then our
uncertainty about a subsystem won't generally be of a Gibbsian type
\citep{maesetal1999}.


We can now calculate the density for $\yl$ given the sequence of sample
activities $\yst$ using Bayes's theorem. We can introduce the relative
frequencies $\yf$ for the sample activities and proceed as in
\sect~\ref{sec:full-pop_data}, obtaining
\begin{equation}
  \label{eq:to_maxent_sample}
  \begin{split}
  \pf(\yst \| \yl, \yI) &= \prod_{t} \pf(\ys_{t} \| \yl, \yI)
    \\ &=
    \prod_{\ys}\bigl[ \pf(\ys \| \yl, \yI) \bigr]^{T f_{\ys}}
    \\ &=
    \exp\Bigl(-T\bigl\{
    \yH\bigl[\yf, \pf(\ys \| \yl, \yI)\bigr]
    -\yH(\yf)\bigr\}\Bigr).
\end{split}
\end{equation}

As in the previous section, if the density $\pf(\yl \| \yI)$ for $\yl$ is
uniform then the mode of $\pf(\yl \| \yst, \yI)$ can be obtained by
minimizing the relative entropy between $\yf$ and $\pf(\ys \| \yl, \yI)$.
In the present case, however, the expression for the latter probability,
\eqn~\eqref{eq:koopman-pitman_likelihood_sample}, does not lead to the
classical maximum-entropy equations. We find instead
\begin{subequations}
  \label{eq:minimize_lagrangian}
  \begin{gather}
    \de_{\yll_{r}}\ln Z(\yl) = \yccs_{r}(\yl)
    \\
    \shortintertext{with}
   \yccs_{r}(\yl) \defd
   \tsum_{\ys} f_{\ys}\; c_{r\ys}(\yl),
   \qquad
    c_{r\ys}(\yl) \defd \frac{
      \sum_{\yS} C_{r\yS}\; G_{\ys\yS}\;\pf(\yS \| \yl, \yI)
    }{
      \sum_{\yS} G_{\ys\yS}\;\pf(\yS \| \yl, \yI)
    }.
  \end{gather}
\end{subequations}
or more explicitly
\begin{equation}
  \label{eq:explicit_eqn_lambda_sample}
  \frac{\sum_{\yS} C_{r\yS}\;\yg_{\yS}
    \exp\bigl( \tsum_{r} \yll_{r} C_{r\yS} \bigr)}{
    \sum_{\yS} \yg_{\yS}
    \exp\bigl( \tsum_{r} \yll_{r} C_{r\yS} \bigr)
  }
  =
  \sum_{\ys} f_{\ys}\;
  \frac{
    \sum_{\yS} C_{r\yS}\; G_{\ys\yS}\;\yg_{\yS}
    \exp\bigl( \tsum_{r} \yll_{r} C_{r\yS} \bigr)
  }{
    \sum_{\yS} G_{\ys\yS}\;\yg_{\yS}
    \exp\bigl( \tsum_{r} \yll_{r} C_{r\yS} \bigr)
  }.
\end{equation}
The expressions~\eqref{eq:minimize_lagrangian} define $r$ functions
$c_{r\ys}(\yl)$ of the sample activity $\ys$ \emph{and of $\yl$} that have
a role analogous to the statistic $C_{r\yS}$ of the full-population
activity $\yS$, in the sense that
\begin{equation}
  \label{eq:statistics_for_sample}
  \begin{split}
  \E[c_{r\ys}(\yl) \| \yl, \yI]
  &=  \E(C_{r\yS} \| \yl, \yI),
  \\
  \sum_{\ys} c_{r\ys}(\yl)\; \pf(\ys \| \yl, \yI)
 &= \sum_{\yS} C_{r\yS}\; \pf(\yS \| \yl, \yI),
\end{split}
\end{equation}
as can be verified by substitution and the
definition~\eqref{eq:dob_sample_given_pop} of $G_{\ys\yS}$.

The mode $\yl$ is given by
\begin{equation}
  \arginf_{\yl}\Bigl\{ 
  \ln Z(\yl) - \sum_{\ys} f_{\ys}\;
  \ln\bigl[\tsum_{\yS}
  G_{\ys\yS}\;\yg_{\yS}\;
  \exp\bigl( \tsum_{r} \yll_{r} C_{r\yS} \bigr)\bigr]
  \Bigr\}.
\end{equation}
Compare this with the standard maximum-entropy case \citep{meadetal1984},
which can be written as
\begin{equation}
  \arginf_{\yl}\Bigl[
  \ln Z(\yl) - \sum_{\yS} F_{\yS}\;
  \ln \exp\bigl( \tsum_{r} \yll_{r} C_{r\yS} \bigr)
  \Bigr].
\end{equation}

\bigskip

\textbf{Important differences from the \enquote{dilemma} paper:}

Consider the following statistics:
\begin{equation}
  \label{eq:statistics_factorial}
  C_{r\yS} \defd \binom{\yS}{r}/\binom{N}{r}
  \qquad r\in{1,2,\dotsc}.
\end{equation}
In particular, $C_{1\yS} = \yS/N$ and $C_{r\yS}$ is the number of
$r$-tuples of simultaneously active neurons divided by the number of
possible ones. Analogous statistics $\binom{\ys}{r}/\binom{n}{r}$ can
be considered for the sample.

These statistics have this special property
\citep{portamanaetal2015,portamanaetal2018b}:
\begin{equation}
  \label{eq:equality_statistics_sample_pop}
  \sum_{\ys} \binom{\ys}{r}/\binom{n}{r}\;p(\ys)
=   \sum_{\yS} \binom{\yS}{r}/\binom{N}{r}\;p(\yS),
\end{equation}
no matter what the \dobs\ about full-population and sample might be,
provided that they are related by $p(\ys) = \sum_{\yS}G_{\ys\yS}\;p(\yS)$.

Note, however, that given $p(\yS)$ and any statistic $C_{r\yS}$ it is
always possible to create a function $c_{r\yS}$ of the sample activity that
satisfies
\begin{equation}
  \label{eq:equality_statistics_general}
  \sum_{\ys} c_{r\ys}\;p(\ys)
  =   \sum_{\yS} C_{r\yS}\;p(\yS)
\end{equation}
namely,
\begin{equation}
  \label{eq:ad-hoc_statistics}
  c_{r\ys} \defd \frac{\sum_{\yS} C_{r\yS}\;G_{\ys\yS}\;p(\yS)}{
    \sum_{\yS}G_{\ys\yS}\;p(\yS)}.
\end{equation}
But this function \emph{depends on the specific p(\yS)}.

Now consider the minimization of the relative entropy in
\eqn~\eqref{eq:to_maxent_sample}, corresponding to
\eqns~\eqref{eq:minimize_lagrangian}, for statistics given
by~\eqref{eq:statistics_factorial}. Remember that the assumption of
sufficiency asymptotically gives zero probability to observing relative
frequencies of full population and sample that lie outside a particular
$R$-dimensional set.

We have two possibilities when $T$ is very large:
\begin{enumerate}[label=\arabic*.]
\item If the frequencies $\yf$ observed for the sample are among those
  admitted by the sufficiency hypothesis, then there is a $\yl^*$ for which
  \begin{equation}
    \label{eq:condition_admitted_freqs}
    f_{\ys} \approx \sum_{\yS} G_{\ys\yS}\;\pf(\yS \| \yl^*, \yI).
  \end{equation}
Moreover, for that $\yl$ we also have
\begin{equation}
  \label{eq:condition_admitted_freqs_expectation}
  \sum_{\ys}  f_{\ys}\;\binom{\ys}{r}/\binom{n}{r}
  \approx
  \sum_{\ys} f_{\ys}\;c_{r\ys}(\yl^*).
\end{equation}
and the result of the relative-entropy minimization is equivalent to
finding the maximum-entropy distribution for the full population with
expectations for the statistics~\eqref{eq:statistics_factorial} constrained
to equal the empirical ones observed in the sample.
\item If the frequencies $\yf$ observed for the sample are \emph{not} among
  those admitted by the sufficiency hypothesis, then the minimization of
  the relative entropy yields a distribution for the full-population that
  does \emph{not} satisfy those constraints.
\end{enumerate}




\iffalse
Result:
\begin{equation}
    \pf(\yl\| \yst, \yI) \propto
    \pf(\yl \| \yI)\;
    \exp\Bigl\{T \sum_{\ys}
    f_{\ys} \ln\Bigl[
    \tsum_{\yS} G_{\ys\yS}\;
    \tfrac{\yg_{\yS}}{Z(\yl)}
    \exp\bigl(\tsum_{r} \yll_r C_{r\yS}\bigr)
    \Bigr]\Bigr\}.
\end{equation}
\fi


\textcolor{white}{If you find this you can claim a postcard from me.}


%%\setlength{\intextsep}{0.5ex}% with wrapfigure
%\begin{figure}[p!]%{r}{0.4\linewidth} % with wrapfigure
%  \centering\includegraphics[trim={12ex 0 18ex 0},clip,width=\linewidth]{maxent_saddle.png}\\
%\caption{caption}\label{fig:comparison_a5}
%\end{figure}% exp_family_maxent.nb


%%%%%%%%%%%%%%%%%%%%%%%%%%%%%%%%%%%%%%%%%%%%%%%%%%%%%%%%%%%%%%%%%%%%%%%%%%%%
%%% Acknowledgements
%%%%%%%%%%%%%%%%%%%%%%%%%%%%%%%%%%%%%%%%%%%%%%%%%%%%%%%%%%%%%%%%%%%%%%%%%%%% 
\iffalse
\begin{acknowledgements}
  \ldots to Mari \amp\ Miri for continuous encouragement and affection, and
  to Buster Keaton and Saitama for filling life with awe and inspiration.
  To the developers and maintainers of \LaTeX, Emacs, AUC\TeX, Open Science
  Framework, Python, Inkscape, Sci-Hub for making a free and unfiltered
  scientific exchange possible.
%\rotatebox{15}{P}\rotatebox{5}{I}\rotatebox{-10}{P}\rotatebox{10}{\reflectbox{P}}\rotatebox{-5}{O}.
\sourceatright{\autanet}
\end{acknowledgements}
\fi

%%%%%%%%%%%%%%%%%%%%%%%%%%%%%%%%%%%%%%%%%%%%%%%%%%%%%%%%%%%%%%%%%%%%%%%%%%%%
%%% Appendices
%%%%%%%%%%%%%%%%%%%%%%%%%%%%%%%%%%%%%%%%%%%%%%%%%%%%%%%%%%%%%%%%%%%%%%%%%%%% 
\newpage
% %\renewcommand*{\appendixpagename}{Appendix}
% %\renewcommand*{\appendixname}{Appendix}
% %\appendixpage
% \appendix

%%%%%%%%%%%%%%%%%%%%%%%%%%%%%%%%%%%%%%%%%%%%%%%%%%%%%%%%%%%%%%%%%%%%%%%%%%%%
%%% Bibliography
%%%%%%%%%%%%%%%%%%%%%%%%%%%%%%%%%%%%%%%%%%%%%%%%%%%%%%%%%%%%%%%%%%%%%%%%%%%% 
\defbibnote{prenote}{{\footnotesize (\enquote{de $X$} is listed under D,
    \enquote{van $X$} under V, and so on, regardless of national
    conventions.)\par}}
% \defbibnote{postnote}{\par\medskip\noindent{\footnotesize% Note:
%     \arxivp \mparcp \philscip \biorxivp}}

\printbibliography[prenote=prenote%,postnote=postnote
]

\end{document}

%%%%%%%%%%%%%%%%%%%%%%%%%%%%%%%%%%%%%%%%%%%%%%%%%%%%%%%%%%%%%%%%%%%%%%%%%%%%
%%% Cut text (won't be compiled)
%%%%%%%%%%%%%%%%%%%%%%%%%%%%%%%%%%%%%%%%%%%%%%%%%%%%%%%%%%%%%%%%%%%%%%%%%%%% 

% \begin{equation}
%   \label{eq:likelihood_via_total_probability}
%   \begin{split}
%     \pf(\yst \| \yl, \yI) &=
% \sum_{\ySt} \pf(\yst \| \ySt, \yl, \yI)\;
% \pf(\ySt \| \yl, \yI),
% \\ &=
% \sum_{\yS_{t_1} \dotso \yS_{t_T}}
% \prod_{t}
% \pf( \ys_t \| yS_t, \yl, \yI)\;
% \frac{\yg_{\yS_{t}}}{Z(\yl)}
% \exp\bigl(\tsum_{r} \yll_r C_{r\yS_{t}}\bigr)
% \\ &=
% \prod_{t}\sum_{\yS}
% G_{\ys_{t}\yS}\;
% \frac{\yg_{\yS}}{Z(\yl)}
% \exp\bigl(\tsum_{r} \yll_r C_{r\yS}\bigr)
% \\ &=
% \prod_{\ys}\biggl[ \sum_{\yS}
% G_{\ys_{t}\yS}\;
% \frac{\yg_{\yS}}{Z(\yl)}
% \exp\bigl(\tsum_{r} \yll_r C_{r\yS}\bigr) \biggr]^{Tf_{\ys}}
% \\ &=
% \exp\biggl\{-T\,
% \yH\biggl[\yf,
% \sum_{\yS}
% G_{\ys_{t}\yS}\;
% \frac{\yg_{\yS}}{Z(\yl)}
% \exp\bigl(\tsum_{r} \yll_r C_{r\yS}\bigr) \biggr]
%  -T\yH(\yf)\biggr\}.
% \end{split}
% \end{equation}

%%% Local Variables: 
%%% mode: LaTeX
%%% TeX-PDF-mode: t
%%% TeX-master: t
%%% End: 
