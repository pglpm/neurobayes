\pdfoutput=1
%% Author: PGL  Porta Mana
%% Created: 2015-05-01T20:53:34+0200
%% Last-Updated: 2019-01-31T22:36:22+0100
%%%%%%%%%%%%%%%%%%%%%%%%%%%%%%%%%%%%%%%%%%%%%%%%%%%%%%%%%%%%%%%%%%%%%%
% Report-no: ***
\newif\ifarxiv
\arxivfalse
\ifarxiv\pdfmapfile{+classico.map}\fi
\newif\ifafour
\afourfalse % true = A4, false = A5
\newif\iftypodisclaim % typographical disclaim on the side
\typodisclaimtrue
\newcommand*{\memfontfamily}{zplx}
\newcommand*{\memfontpack}{newpxtext}
\documentclass[\ifafour a4paper,12pt,\else a5paper,10pt,\fi%extrafontsizes,%
onecolumn,oneside,article,%french,italian,german,swedish,latin,
british%
]{memoir}
\newcommand*{\updated}{\today}
\newcommand*{\firstdraft}{25 January 2019}
\newcommand*{\firstpublished}{***}
\newcommand*{\propertitle}{An overview of inferences\\about neuronal
networks}
\newcommand*{\pdftitle}{An overview of inferences about neuronal networks}
\newcommand*{\headtitle}{Inferences for neuronal networks}
\newcommand*{\pdfauthor}{P.G.L.  Porta Mana}
\newcommand*{\headauthor}{Porta Mana}
\newcommand*{\reporthead}{}
%%%%%%%%%%%%%%%%%%%%%%%%%%%%%%%%%%%%%%%%%%%%%%%%%%%%%%%%%%%%%%%%%%%%%%%%%%%%
%%%%%%%%%%%%%%%%%%%%%%%%%%%%%%%%%%%%%%%%%%%%%%%%%%%%%%%%%%%%%%%%%%%%%%%%%%%%
%\usepackage{pifont}
%\usepackage{fontawesome}
\usepackage[T1]{fontenc} 
\input{glyphtounicode} \pdfgentounicode=1
\usepackage[utf8]{inputenx}
%\usepackage{newunicodechar}
% \newunicodechar{Ĕ}{\u{E}}
% \newunicodechar{ĕ}{\u{e}}
% \newunicodechar{Ĭ}{\u{I}}
% \newunicodechar{ĭ}{\u{\i}}
% \newunicodechar{Ŏ}{\u{O}}
% \newunicodechar{ŏ}{\u{o}}
% \newunicodechar{Ŭ}{\u{U}}
% \newunicodechar{ŭ}{\u{u}}
% \newunicodechar{Ā}{\=A}
% \newunicodechar{ā}{\=a}
% \newunicodechar{Ē}{\=E}
% \newunicodechar{ē}{\=e}
% \newunicodechar{Ī}{\=I}
% \newunicodechar{ī}{\={\i}}
% \newunicodechar{Ō}{\=O}
% \newunicodechar{ō}{\=o}
% \newunicodechar{Ū}{\=U}
% \newunicodechar{ū}{\=u}
% \newunicodechar{Ȳ}{\=Y}
% \newunicodechar{ȳ}{\=y}

\newcommand*{\bmmax}{0} % reduce number of bold fonts, before bm
\newcommand*{\hmmax}{0} % reduce number of heavy fonts, before bm
\usepackage{textcomp}
\usepackage[normalem]{ulem}
% \makeatletter
% \def\ssout{\bgroup \ULdepth=-.35ex%\UL@setULdepth
%  \markoverwith{\lower\ULdepth\hbox
%    {\kern-.03em\vbox{\hrule width.2em\kern1.2\p@\hrule}\kern-.03em}}%
%  \ULon}
% \makeatother
\usepackage{amsmath}
\usepackage{mathtools}
\addtolength{\jot}{\jot} % increase spacing in multiline formulae
\usepackage{empheq}% automatically calls amsmath and mathtools
\newcommand*{\widefbox}[1]{\fbox{\hspace{1em}#1\hspace{1em}}}
\setlength{\multlinegap}{0pt}
%\usepackage{fancybox}
%\usepackage{framed}
% \usepackage[misc]{ifsym} % for dice
% \newcommand*{\diceone}{{\scriptsize\Cube{1}}}
\usepackage{amssymb}
\usepackage{amsxtra}

\usepackage[main=british,french,italian,german,swedish,latin,esperanto]{babel}\selectlanguage{british}
\newcommand*{\langfrench}{\foreignlanguage{french}}
\newcommand*{\langgerman}{\foreignlanguage{german}}
\newcommand*{\langitalian}{\foreignlanguage{italian}}
\newcommand*{\langswedish}{\foreignlanguage{swedish}}
\newcommand*{\langlatin}{\foreignlanguage{latin}}
\newcommand*{\langnohyph}{\foreignlanguage{nohyphenation}}

\usepackage[autostyle=false,autopunct=false,english=british]{csquotes}
\setquotestyle{british}

\usepackage{amsthm}
\newcommand*{\QED}{\textsc{q.e.d.}}
\renewcommand*{\qedsymbol}{\QED}
\theoremstyle{remark}
\newtheorem{note}{Note}
\newtheorem*{remark}{Note}
\newtheoremstyle{innote}{\parsep}{\parsep}{\footnotesize}{}{}{}{0pt}{}
\theoremstyle{innote}
\newtheorem*{innote}{}


\usepackage[shortlabels,inline]{enumitem}
\SetEnumitemKey{para}{itemindent=\parindent,leftmargin=0pt,listparindent=\parindent,parsep=0pt,itemsep=\topsep}
% \begin{asparaenum} = \begin{enumerate}[para]
% \begin{inparaenum} = \begin{enumerate*}
\setlist[enumerate,2]{label=\alph*.}
\setlist[enumerate]{label=\arabic*.,leftmargin=1.5\parindent}
\setlist[itemize]{leftmargin=1.5\parindent}
\setlist[description]{leftmargin=1.5\parindent}
% old alternative:
% \setlist[enumerate,2]{label=\alph*.}
% \setlist[enumerate]{leftmargin=\parindent}
% \setlist[itemize]{leftmargin=\parindent}
% \setlist[description]{leftmargin=\parindent}

\usepackage[babel,theoremfont,largesc]{newpxtext}
\usepackage[bigdelims,nosymbolsc%,smallerops % probably arXiv doesn't have it
]{newpxmath}
\useosf\linespread{1.083}
%% smaller operators for old version of newpxmath
\makeatletter
\def\re@DeclareMathSymbol#1#2#3#4{%
    \let#1=\undefined
    \DeclareMathSymbol{#1}{#2}{#3}{#4}}
%\re@DeclareMathSymbol{\bigsqcupop}{\mathop}{largesymbols}{"46}
%\re@DeclareMathSymbol{\bigodotop}{\mathop}{largesymbols}{"4A}
\re@DeclareMathSymbol{\bigoplusop}{\mathop}{largesymbols}{"4C}
\re@DeclareMathSymbol{\bigotimesop}{\mathop}{largesymbols}{"4E}
\re@DeclareMathSymbol{\sumop}{\mathop}{largesymbols}{"50}
\re@DeclareMathSymbol{\prodop}{\mathop}{largesymbols}{"51}
\re@DeclareMathSymbol{\bigcupop}{\mathop}{largesymbols}{"53}
\re@DeclareMathSymbol{\bigcapop}{\mathop}{largesymbols}{"54}
%\re@DeclareMathSymbol{\biguplusop}{\mathop}{largesymbols}{"55}
\re@DeclareMathSymbol{\bigwedgeop}{\mathop}{largesymbols}{"56}
\re@DeclareMathSymbol{\bigveeop}{\mathop}{largesymbols}{"57}
%\re@DeclareMathSymbol{\bigcupdotop}{\mathop}{largesymbols}{"DF}
%\re@DeclareMathSymbol{\bigcapplusop}{\mathop}{largesymbolsPXA}{"00}
%\re@DeclareMathSymbol{\bigsqcupplusop}{\mathop}{largesymbolsPXA}{"02}
%\re@DeclareMathSymbol{\bigsqcapplusop}{\mathop}{largesymbolsPXA}{"04}
%\re@DeclareMathSymbol{\bigsqcapop}{\mathop}{largesymbolsPXA}{"06}
\re@DeclareMathSymbol{\bigtimesop}{\mathop}{largesymbolsPXA}{"10}
%\re@DeclareMathSymbol{\coprodop}{\mathop}{largesymbols}{"60}
%\re@DeclareMathSymbol{\varprod}{\mathop}{largesymbolsPXA}{16}
\makeatother


%% With euler font cursive for Greek letters - the [1] means 100% scaling
\DeclareFontFamily{U}{egreek}{\skewchar\font'177}%
\DeclareFontShape{U}{egreek}{m}{n}{<-6>s*[1]eurm5 <6-8>s*[1]eurm7 <8->s*[1]eurm10}{}%
\DeclareFontShape{U}{egreek}{m}{it}{<->s*[1]eurmo10}{}%
\DeclareFontShape{U}{egreek}{b}{n}{<-6>s*[1]eurb5 <6-8>s*[1]eurb7 <8->s*[1]eurb10}{}%
\DeclareFontShape{U}{egreek}{b}{it}{<->s*[1]eurbo10}{}%
\DeclareSymbolFont{egreeki}{U}{egreek}{m}{it}%
\SetSymbolFont{egreeki}{bold}{U}{egreek}{b}{it}% from the amsfonts package
\DeclareSymbolFont{egreekr}{U}{egreek}{m}{n}%
\SetSymbolFont{egreekr}{bold}{U}{egreek}{b}{n}% from the amsfonts package
% Take also \sum, \prod, \coprod symbols from Euler fonts
\DeclareFontFamily{U}{egreekx}{\skewchar\font'177}
\DeclareFontShape{U}{egreekx}{m}{n}{%
       <-7.5>s*[0.9]euex7%
    <7.5-8.5>s*[0.9]euex8%
    <8.5-9.5>s*[0.9]euex9%
    <9.5->s*[0.9]euex10%
}{}
\DeclareSymbolFont{egreekx}{U}{egreekx}{m}{n}
\DeclareMathSymbol{\sumop}{\mathop}{egreekx}{"50}
\DeclareMathSymbol{\prodop}{\mathop}{egreekx}{"51}
\DeclareMathSymbol{\coprodop}{\mathop}{egreekx}{"60}
\makeatletter
\def\sum{\DOTSI\sumop\slimits@}
\def\prod{\DOTSI\prodop\slimits@}
\def\coprod{\DOTSI\coprodop\slimits@}
\makeatother
\ifarxiv\else\input{../definegreek.tex}\fi% make sure no CMF greek letters sneak in
% Greek letters not usually given in LaTeX. Comment the unneeded ones
% \DeclareMathSymbol{\varpartial}{\mathalpha}{egreeki}{"40}
% \DeclareMathSymbol{\partialup}{\mathalpha}{egreekr}{"40}
% \DeclareMathSymbol{\alpha}{\mathalpha}{egreeki}{"0B}
% \DeclareMathSymbol{\beta}{\mathalpha}{egreeki}{"0C}
% \DeclareMathSymbol{\gamma}{\mathalpha}{egreeki}{"0D}
% \DeclareMathSymbol{\delta}{\mathalpha}{egreeki}{"0E}
% \DeclareMathSymbol{\epsilon}{\mathalpha}{egreeki}{"0F}
% \DeclareMathSymbol{\zeta}{\mathalpha}{egreeki}{"10}
% \DeclareMathSymbol{\eta}{\mathalpha}{egreeki}{"11}
% \DeclareMathSymbol{\theta}{\mathalpha}{egreeki}{"12}
% \DeclareMathSymbol{\iota}{\mathalpha}{egreeki}{"13}
% \DeclareMathSymbol{\kappa}{\mathalpha}{egreeki}{"14}
% \DeclareMathSymbol{\lambda}{\mathalpha}{egreeki}{"15}
% \DeclareMathSymbol{\mu}{\mathalpha}{egreeki}{"16}
% \DeclareMathSymbol{\nu}{\mathalpha}{egreeki}{"17}
% \DeclareMathSymbol{\xi}{\mathalpha}{egreeki}{"18}
% \DeclareMathSymbol{\omicron}{\mathalpha}{egreeki}{"6F}
% \DeclareMathSymbol{\pi}{\mathalpha}{egreeki}{"19}
% \DeclareMathSymbol{\rho}{\mathalpha}{egreeki}{"1A}
% \DeclareMathSymbol{\sigma}{\mathalpha}{egreeki}{"1B}
% \DeclareMathSymbol{\tau}{\mathalpha}{egreeki}{"1C}
% \DeclareMathSymbol{\upsilon}{\mathalpha}{egreeki}{"1D}
% \DeclareMathSymbol{\phi}{\mathalpha}{egreeki}{"1E}
% \DeclareMathSymbol{\chi}{\mathalpha}{egreeki}{"1F}
% \DeclareMathSymbol{\psi}{\mathalpha}{egreeki}{"20}
% \DeclareMathSymbol{\omega}{\mathalpha}{egreeki}{"21}
% \DeclareMathSymbol{\varepsilon}{\mathalpha}{egreeki}{"22}
% \DeclareMathSymbol{\vartheta}{\mathalpha}{egreeki}{"23}
% \DeclareMathSymbol{\varpi}{\mathalpha}{egreeki}{"24}
% \let\varrho\rho 
% \let\varsigma\sigma
% \let\varkappa\kappa
% \DeclareMathSymbol{\varphi}{\mathalpha}{egreeki}{"27}
% %
% \DeclareMathSymbol{\varAlpha}{\mathalpha}{egreeki}{"41}
% \DeclareMathSymbol{\varBeta}{\mathalpha}{egreeki}{"42}
% \DeclareMathSymbol{\varGamma}{\mathalpha}{egreeki}{"00}
% \DeclareMathSymbol{\varDelta}{\mathalpha}{egreeki}{"01}
% \DeclareMathSymbol{\varEpsilon}{\mathalpha}{egreeki}{"45}
% \DeclareMathSymbol{\varZeta}{\mathalpha}{egreeki}{"5A}
% \DeclareMathSymbol{\varEta}{\mathalpha}{egreeki}{"48}
% \DeclareMathSymbol{\varTheta}{\mathalpha}{egreeki}{"02}
% \DeclareMathSymbol{\varIota}{\mathalpha}{egreeki}{"49}
% \DeclareMathSymbol{\varKappa}{\mathalpha}{egreeki}{"4B}
% \DeclareMathSymbol{\varLambda}{\mathalpha}{egreeki}{"03}
% \DeclareMathSymbol{\varMu}{\mathalpha}{egreeki}{"4D}
% \DeclareMathSymbol{\varNu}{\mathalpha}{egreeki}{"4E}
% \DeclareMathSymbol{\varXi}{\mathalpha}{egreeki}{"04}
% \DeclareMathSymbol{\varOmicron}{\mathalpha}{egreeki}{"4F}
% \DeclareMathSymbol{\varPi}{\mathalpha}{egreeki}{"05}
% \DeclareMathSymbol{\varRho}{\mathalpha}{egreeki}{"50}
% \DeclareMathSymbol{\varSigma}{\mathalpha}{egreeki}{"06}
% \DeclareMathSymbol{\varTau}{\mathalpha}{egreeki}{"54}
% \DeclareMathSymbol{\varUpsilon}{\mathalpha}{egreeki}{"07}
% \DeclareMathSymbol{\varPhi}{\mathalpha}{egreeki}{"08}
% \DeclareMathSymbol{\varChi}{\mathalpha}{egreeki}{"58}
% \DeclareMathSymbol{\varPsi}{\mathalpha}{egreeki}{"09}
% \DeclareMathSymbol{\varOmega}{\mathalpha}{egreeki}{"0A} 
% %
% \DeclareMathSymbol{\Alpha}{\mathalpha}{egreekr}{"41}
% \DeclareMathSymbol{\Beta}{\mathalpha}{egreekr}{"42}
% \DeclareMathSymbol{\Gamma}{\mathalpha}{egreekr}{"00}
% \DeclareMathSymbol{\Delta}{\mathalpha}{egreekr}{"01}
% \DeclareMathSymbol{\Epsilon}{\mathalpha}{egreekr}{"45}
% \DeclareMathSymbol{\Zeta}{\mathalpha}{egreekr}{"5A}
% \DeclareMathSymbol{\Eta}{\mathalpha}{egreekr}{"48}
% \DeclareMathSymbol{\Theta}{\mathalpha}{egreekr}{"02}
% \DeclareMathSymbol{\Iota}{\mathalpha}{egreekr}{"49}
% \DeclareMathSymbol{\Kappa}{\mathalpha}{egreekr}{"4B}
% \DeclareMathSymbol{\Lambda}{\mathalpha}{egreekr}{"03}
% \DeclareMathSymbol{\Mu}{\mathalpha}{egreekr}{"4D}
% \DeclareMathSymbol{\Nu}{\mathalpha}{egreekr}{"4E}
% \DeclareMathSymbol{\Xi}{\mathalpha}{egreekr}{"04}
% \DeclareMathSymbol{\Omicron}{\mathalpha}{egreekr}{"4F}
% \DeclareMathSymbol{\Pi}{\mathalpha}{egreekr}{"05}
% \DeclareMathSymbol{\Rho}{\mathalpha}{egreekr}{"50}
% \DeclareMathSymbol{\Sigma}{\mathalpha}{egreekr}{"06}
% \DeclareMathSymbol{\Tau}{\mathalpha}{egreekr}{"54}
% \DeclareMathSymbol{\Upsilon}{\mathalpha}{egreekr}{"07}
% \DeclareMathSymbol{\Phi}{\mathalpha}{egreekr}{"08}
% \DeclareMathSymbol{\Chi}{\mathalpha}{egreekr}{"58}
% \DeclareMathSymbol{\Psi}{\mathalpha}{egreekr}{"09}
% \DeclareMathSymbol{\Omega}{\mathalpha}{egreekr}{"0A}
% %
% \DeclareMathSymbol{\alphaup}{\mathalpha}{egreekr}{"0B}
% \DeclareMathSymbol{\betaup}{\mathalpha}{egreekr}{"0C}
% \DeclareMathSymbol{\gammaup}{\mathalpha}{egreekr}{"0D}
% \DeclareMathSymbol{\deltaup}{\mathalpha}{egreekr}{"0E}
% \DeclareMathSymbol{\epsilonup}{\mathalpha}{egreekr}{"0F}
% \DeclareMathSymbol{\zetaup}{\mathalpha}{egreekr}{"10}
% \DeclareMathSymbol{\etaup}{\mathalpha}{egreekr}{"11}
% \DeclareMathSymbol{\thetaup}{\mathalpha}{egreekr}{"12}
% \DeclareMathSymbol{\iotaup}{\mathalpha}{egreekr}{"13}
% \DeclareMathSymbol{\kappaup}{\mathalpha}{egreekr}{"14}
% \DeclareMathSymbol{\lambdaup}{\mathalpha}{egreekr}{"15}
% \DeclareMathSymbol{\muup}{\mathalpha}{egreekr}{"16}
% \DeclareMathSymbol{\nuup}{\mathalpha}{egreekr}{"17}
% \DeclareMathSymbol{\xiup}{\mathalpha}{egreekr}{"18}
% \DeclareMathSymbol{\omicronup}{\mathalpha}{egreekr}{"6F}
%  \DeclareMathSymbol{\piup}{\mathalpha}{egreekr}{"19}
% \DeclareMathSymbol{\rhoup}{\mathalpha}{egreekr}{"1A}
% \DeclareMathSymbol{\sigmaup}{\mathalpha}{egreekr}{"1B}
% \DeclareMathSymbol{\tauup}{\mathalpha}{egreekr}{"1C}
% \DeclareMathSymbol{\upsilonup}{\mathalpha}{egreekr}{"1D}
% \DeclareMathSymbol{\phiup}{\mathalpha}{egreekr}{"1E}
% \DeclareMathSymbol{\chiup}{\mathalpha}{egreekr}{"1F}
% \DeclareMathSymbol{\psiup}{\mathalpha}{egreekr}{"20}
% \DeclareMathSymbol{\omegaup}{\mathalpha}{egreekr}{"21}
% \DeclareMathSymbol{\varepsilonup}{\mathalpha}{egreekr}{"22}
% \DeclareMathSymbol{\varthetaup}{\mathalpha}{egreekr}{"23}
% \DeclareMathSymbol{\varpiup}{\mathalpha}{egreekr}{"24}
% \let\varrhoup\rhoup 
% \let\varsigmaup\sigmaup
% \let\varkappaup\kappaup
% \DeclareMathSymbol{\varphiup}{\mathalpha}{egreekr}{"27}

% Optima as sans-serif font
%\usepackage%[scaled=0.9]%
%{classico}
\renewcommand\sfdefault{uop}
\DeclareMathAlphabet{\mathsf}  {T1}{\sfdefault}{m}{sl}
\SetMathAlphabet{\mathsf}{bold}{T1}{\sfdefault}{b}{sl}
\newcommand*{\mathte}[1]{\textbf{\textit{\textsf{#1}}}}
% Upright sans-serif math alphabet
% \DeclareMathAlphabet{\mathsu}  {T1}{\sfdefault}{m}{n}
% \SetMathAlphabet{\mathsu}{bold}{T1}{\sfdefault}{b}{n}

% DejaVu Mono as typewriter text
\usepackage[scaled=0.84]{DejaVuSansMono}


\usepackage{mathdots}

\usepackage[usenames]{xcolor}
% Tol (2012) colour-blind-, print-, screen-friendly colours, alternative scheme; Munsell terminology
\definecolor{mypurpleblue}{RGB}{68,119,170}
\definecolor{myblue}{RGB}{102,204,238}
\definecolor{mygreen}{RGB}{34,136,51}
\definecolor{myyellow}{RGB}{204,187,68}
\definecolor{myred}{RGB}{238,102,119}
\definecolor{myredpurple}{RGB}{170,51,119}
\definecolor{mygrey}{RGB}{187,187,187}

% Tol (2012) colour-blind-, print-, screen-friendly colours; Munsell terminology
% \definecolor{lbpurple}{RGB}{51,34,136}
% \definecolor{lblue}{RGB}{136,204,238}
% \definecolor{lbgreen}{RGB}{68,170,153}
% \definecolor{lgreen}{RGB}{17,119,51}
% \definecolor{lgyellow}{RGB}{153,153,51}
% \definecolor{lyellow}{RGB}{221,204,119}
% \definecolor{lred}{RGB}{204,102,119}
% \definecolor{lpred}{RGB}{136,34,85}
% \definecolor{lrpurple}{RGB}{170,68,153}
 \definecolor{lgrey}{RGB}{221,221,221}
%\newcommand*\mycolourbox[1]{%
%\colorbox{mygrey}{\hspace{1em}#1\hspace{1em}}}
\colorlet{shadecolor}{lgrey}

\usepackage{bm}
\usepackage{microtype}

\usepackage[backend=biber,mcite,%subentry,
citestyle=authoryear-comp,bibstyle=pglpm-authoryear,autopunct=false,sorting=ny,sortcites=false,natbib=false,maxcitenames=1,maxbibnames=8,minbibnames=8,giveninits=true,uniquename=false,uniquelist=false,maxalphanames=1,block=space,hyperref=true,defernumbers=false,useprefix=true,sortupper=false,language=british,parentracker=false]{biblatex}
\DeclareSortingScheme{ny}{\sort{\field{sortname}\field{author}\field{editor}}\sort{\field{year}}}
\iffalse\makeatletter%%% replace parenthesis with brackets
\newrobustcmd*{\parentexttrack}[1]{%
  \begingroup
  \blx@blxinit
  \blx@setsfcodes
  \blx@bibopenparen#1\blx@bibcloseparen
  \endgroup}
\AtEveryCite{%
  \let\parentext=\parentexttrack%
  \let\bibopenparen=\bibopenbracket%
  \let\bibcloseparen=\bibclosebracket}
\makeatother\fi
\DefineBibliographyExtras{british}{\def\finalandcomma{\addcomma}}
\renewcommand*{\finalnamedelim}{\addcomma\space}
\setcounter{biburlnumpenalty}{1}
\setcounter{biburlucpenalty}{0}
\setcounter{biburllcpenalty}{1}
\DeclareDelimFormat{multicitedelim}{\addsemicolon\space}
\DeclareDelimFormat{compcitedelim}{\addsemicolon\space}
\DeclareDelimFormat{postnotedelim}{\space}
\ifarxiv\else\addbibresource{../portamanabib.bib}\fi
\renewcommand{\bibfont}{\footnotesize}
%\appto{\citesetup}{\footnotesize}% smaller font for citations
\defbibheading{bibliography}[\bibname]{\section*{#1}\addcontentsline{toc}{section}{#1}%\markboth{#1}{#1}
}
\newcommand*{\citep}{\parencites}
\newcommand*{\citey}{\parencites*}
%\renewcommand*{\cite}{\parencite}
\renewcommand*{\cites}{\parencites}
\providecommand{\href}[2]{#2}
\providecommand{\eprint}[2]{\texttt{\href{#1}{#2}}}
\newcommand*{\amp}{\&}
% \newcommand*{\citein}[2][]{\textnormal{\textcite[#1]{#2}}%\addtocategory{extras}{#2}
% }
\newcommand*{\citein}[2][]{\textnormal{\textcite[#1]{#2}}%\addtocategory{extras}{#2}
}
\newcommand*{\citebi}[2][]{\textcite[#1]{#2}%\addtocategory{extras}{#2}
}
\newcommand*{\subtitleproc}[1]{}
\newcommand*{\chapb}{ch.}

% \def\arxivp{}
% \def\mparcp{}
% \def\philscip{}
% \def\biorxivp{}
% \newcommand*{\arxivsi}{\texttt{arXiv} eprints available at \url{http://arxiv.org/}.\\}
% \newcommand*{\mparcsi}{\texttt{mp\_arc} eprints available at \url{http://www.ma.utexas.edu/mp_arc/}.\\}
% \newcommand*{\philscisi}{\texttt{philsci} eprints available at \url{http://philsci-archive.pitt.edu/}.\\}
% \newcommand*{\biorxivsi}{\texttt{bioRxiv} eprints available at \url{http://biorxiv.org/}.\\}
\newcommand*{\arxiveprint}[1]{%\global\def\arxivp{\arxivsi}%\citeauthor{0arxivcite}\addtocategory{ifarchcit}{0arxivcite}%eprint
\texttt{\urlalt{https://arxiv.org/abs/#1}{arXiv:\hspace{0pt}#1}}%
%\texttt{\href{http://arxiv.org/abs/#1}{\protect\url{arXiv:#1}}}%
%\renewcommand{\arxivnote}{\texttt{arXiv} eprints available at \url{http://arxiv.org/}.}
}
\newcommand*{\mparceprint}[1]{%\global\def\mparcp{\mparcsi}%\citeauthor{0mparccite}\addtocategory{ifarchcit}{0mparccite}%eprint
\texttt{\urlalt{http://www.ma.utexas.edu/mp_arc-bin/mpa?yn=#1}{mp\_arc:\hspace{0pt}#1}}%
%\texttt{\href{http://www.ma.utexas.edu/mp_arc-bin/mpa?yn=#1}{\protect\url{mp_arc:#1}}}%
%\providecommand{\mparcnote}{\texttt{mp_arc} eprints available at \url{http://www.ma.utexas.edu/mp_arc/}.}
}
\newcommand*{\philscieprint}[1]{%\global\def\philscip{\philscisi}%\citeauthor{0philscicite}\addtocategory{ifarchcit}{0philscicite}%eprint
\texttt{\urlalt{http://philsci-archive.pitt.edu/archive/#1}{PhilSci:\hspace{0pt}#1}}%
%\texttt{\href{http://philsci-archive.pitt.edu/archive/#1}{\protect\url{PhilSci:#1}}}%
%\providecommand{\mparcnote}{\texttt{philsci} eprints available at \url{http://philsci-archive.pitt.edu/}.}
}
\newcommand*{\biorxiveprint}[1]{%\global\def\biorxivp{\biorxivsi}%\citeauthor{0arxivcite}\addtocategory{ifarchcit}{0arxivcite}%eprint
\texttt{\urlalt{https://doi.org/10.1101/#1}{bioRxiv doi:\hspace{0pt}10.1101/#1}}%
%\texttt{\href{http://arxiv.org/abs/#1}{\protect\url{arXiv:#1}}}%
%\renewcommand{\arxivnote}{\texttt{arXiv} eprints available at \url{http://arxiv.org/}.}
}
\newcommand*{\osfeprint}[1]{%
\texttt{\urlalt{https://doi.org/10.17605/osf.io/#1}{Open Science Framework doi:10.17605/osf.io/#1}}%
}

\usepackage{graphicx}
%\usepackage{wrapfig}
%\usepackage{tikz-cd}

\PassOptionsToPackage{hyphens}{url}\usepackage[hypertexnames=false]{hyperref}
\usepackage[depth=4]{bookmark}
\hypersetup{colorlinks=true,bookmarksnumbered,pdfborder={0 0 0.25},citebordercolor={0.2667 0.4667 0.6667},citecolor=mypurpleblue,linkbordercolor={0.6667 0.2 0.4667},linkcolor=myredpurple,urlbordercolor={0.1333 0.5333 0.2},urlcolor=mygreen,breaklinks=true,pdftitle={\pdftitle},pdfauthor={\pdfauthor}}
% \usepackage[vertfit=local]{breakurl}% only for arXiv
\providecommand*{\urlalt}{\href}

%%% Layout. I do not know on which kind of paper the reader will print the
%%% paper on (A4? letter? one-sided? double-sided?). So I choose A5, which
%%% provides a good layout for reading on screen and save paper if printed
%%% two pages per sheet. Average length line is 66 characters and page
%%% numbers are centred.
\ifafour\setstocksize{297mm}{210mm}%{*}% A4
\else\setstocksize{210mm}{5.5in}%{*}% 210x139.7
\fi
\settrimmedsize{\stockheight}{\stockwidth}{*}
\setlxvchars[\normalfont] %313.3632pt for a 66-characters line
\setxlvchars[\normalfont]
\setlength{\trimtop}{0pt}
\setlength{\trimedge}{\stockwidth}
\addtolength{\trimedge}{-\paperwidth}
% The length of the normalsize alphabet is 133.05988pt - 10 pt = 26.1408pc
% The length of the normalsize alphabet is 159.6719pt - 12pt = 30.3586pc
% Bringhurst gives 32pc as boundary optimal with 69 ch per line
% The length of the normalsize alphabet is 191.60612pt - 14pt = 35.8634pc
\ifafour\settypeblocksize{*}{32pc}{1.618} % A4
%\setulmargins{*}{*}{1.667}%gives 5/3 margins % 2 or 1.667
\else\settypeblocksize{*}{26pc}{1.618}% nearer to a 66-line newpx and preserves GR
\fi
\setulmargins{*}{*}{1}%gives equal margins
\setlrmargins{*}{*}{*}
\setheadfoot{\onelineskip}{2.5\onelineskip}
\setheaderspaces{*}{2\onelineskip}{*}
\setmarginnotes{2ex}{10mm}{0pt}
\checkandfixthelayout[nearest]
\fixpdflayout
%%% End layout
%% this fixes missing white spaces
\pdfmapline{+dummy-space <dummy-space.pfb}\pdfinterwordspaceon%

%%% Sectioning
\newcommand*{\asudedication}[1]{%
{\par\centering\textit{#1}\par}}
\newenvironment{acknowledgements}{\section*{Thanks}\addcontentsline{toc}{section}{Thanks}}{\par}
\makeatletter\renewcommand{\appendix}{\par
  \bigskip{\centering
   \interlinepenalty \@M
   \normalfont
   \printchaptertitle{\sffamily\appendixpagename}\par}
  \setcounter{section}{0}%
  \gdef\@chapapp{\appendixname}%
  \gdef\thesection{\@Alph\c@section}%
  \anappendixtrue}\makeatother
\counterwithout{section}{chapter}
\setsecnumformat{\upshape\csname the#1\endcsname\quad}
\setsecheadstyle{\large\bfseries\sffamily%
\raggedright}
\setsubsecheadstyle{\bfseries\sffamily%
\raggedright}
%\setbeforesecskip{-1.5ex plus 1ex minus .2ex}% plus 1ex minus .2ex}
%\setaftersecskip{1.3ex plus .2ex }% plus 1ex minus .2ex}
%\setsubsubsecheadstyle{\bfseries\sffamily\slshape\raggedright}
%\setbeforesubsecskip{1.25ex plus 1ex minus .2ex }% plus 1ex minus .2ex}
%\setaftersubsecskip{-1em}%{-0.5ex plus .2ex}% plus 1ex minus .2ex}
\setsubsecindent{0pt}%0ex plus 1ex minus .2ex}
\setparaheadstyle{\bfseries\sffamily%
\raggedright}
\setcounter{secnumdepth}{2}
\setlength{\headwidth}{\textwidth}
\newcommand{\addchap}[1]{\chapter*[#1]{#1}\addcontentsline{toc}{chapter}{#1}}
\newcommand{\addsec}[1]{\section*{#1}\addcontentsline{toc}{section}{#1}}
\newcommand{\addsubsec}[1]{\subsection*{#1}\addcontentsline{toc}{subsection}{#1}}
\newcommand{\addpara}[1]{\paragraph*{#1.}\addcontentsline{toc}{subsubsection}{#1}}
\newcommand{\addparap}[1]{\paragraph*{#1}\addcontentsline{toc}{subsubsection}{#1}}

% Headers and footers
\copypagestyle{manaart}{plain}
\makeheadrule{manaart}{\headwidth}{0.5\normalrulethickness}
\makeoddhead{manaart}{%
{\footnotesize%\sffamily%
\scshape\headauthor}}{}{{\footnotesize\sffamily%
\headtitle}}
\makeoddfoot{manaart}{}{\thepage}{}
\newcommand*\autanet{\includegraphics[height=\heightof{M}]{autanet.pdf}}
\definecolor{mygray}{gray}{0.333}
\iftypodisclaim%
\ifafour\newcommand\addprintnote{\begin{picture}(0,0)%
\put(245,149){\makebox(0,0){\rotatebox{90}{\tiny\color{mygray}\textsf{This
            document is designed for screen reading and
            two-up printing on A4 or Letter paper}}}}%
\end{picture}}% A4
\else\newcommand\addprintnote{\begin{picture}(0,0)%
\put(176,112){\makebox(0,0){\rotatebox{90}{\tiny\color{mygray}\textsf{This
            document is designed for screen reading and
            two-up printing on A4 or Letter paper}}}}%
\end{picture}}\fi%afourtrue
\makeoddfoot{plain}{}{\makebox[0pt]{\thepage}\addprintnote}{}
\else
\makeoddfoot{plain}{}{\makebox[0pt]{\thepage}}{}
\fi%typodisclaimtrue
\makeoddhead{plain}{}{}{\footnotesize\reporthead}

% \copypagestyle{manainitial}{plain}
% \makeheadrule{manainitial}{\headwidth}{0.5\normalrulethickness}
% \makeoddhead{manainitial}{%
% \footnotesize\sffamily%
% \scshape\headauthor}{}{\footnotesize\sffamily%
% \headtitle}
% \makeoddfoot{manaart}{}{\thepage}{}

\pagestyle{manaart}

\setlength{\droptitle}{-3.9\onelineskip}
\pretitle{\begin{center}\LARGE\sffamily%
\bfseries}
\posttitle{\bigskip\end{center}}

\makeatletter\newcommand*{\atf}{\includegraphics[%trim=1pt 1pt 0pt 0pt,
totalheight=\heightof{@}]{../atblack.png}}\makeatother
\providecommand{\affiliation}[1]{\textsl{\textsf{\footnotesize #1}}}
\providecommand{\epost}[1]{\texttt{\footnotesize\textless#1\textgreater}}
\providecommand{\email}[2]{\href{mailto:#1ZZ@#2 ((remove ZZ))}{#1\protect\atf#2}}

\preauthor{\vspace{-0.5\baselineskip}\begin{center}
\normalsize\sffamily%
\lineskip  0.5em}
\postauthor{\par\end{center}}
\predate{\DTMsetdatestyle{mydate}\begin{center}\footnotesize}
\postdate{\end{center}\vspace{-\medskipamount}}
\usepackage[british]{datetime2}
\DTMnewdatestyle{mydate}%
{% definitions
\renewcommand*{\DTMdisplaydate}[4]{%
\number##3\ \DTMenglishmonthname{##2} ##1}%
\renewcommand*{\DTMDisplaydate}{\DTMdisplaydate}%
}
\DTMsetdatestyle{mydate}


\setfloatadjustment{figure}{\footnotesize}
\captiondelim{\quad}
\captionnamefont{\footnotesize\sffamily%
}
\captiontitlefont{\footnotesize}
\firmlists*
\midsloppy

% handling orphan/widow lines, memman.pdf
% \clubpenalty=10000
% \widowpenalty=10000
% \raggedbottom
% Downes, memman.pdf
\clubpenalty=9996
\widowpenalty=9999
\brokenpenalty=4991
\predisplaypenalty=10000
\postdisplaypenalty=1549
\displaywidowpenalty=1602

\selectlanguage{british}\frenchspacing
%%%%%%%%%%%%%%%%%%%%%%%%%%%%%%%%%%%%%%%%%%%%%%%%%%%%%%%%%%%%%%%%%%%%%%%%%%%%
%%%%%%%%%%%%%%%%%%%%%%%%%%%%%%%%%%%%%%%%%%%%%%%%%%%%%%%%%%%%%%%%%%%%%%%%%%%%
%%%% Paper's details %%%%
\title{\propertitle%\\
%  {\large A geometric commentary on maximum-entropy proofs}% ***
}
\author{%
\hspace*{\stretch{1}}%
%% uncomment if additional authors present
% \parbox{0.5\linewidth}%\makebox[0pt][c]%
% {\protect\centering ***\\%
% \footnotesize\epost{\email{***}{***}}}%
% \hspace*{\stretch{1}}%
\parbox{0.5\linewidth}%\makebox[0pt][c]%
{\protect\centering Luca %
\footnotesize\epost{\email{piero.mana}{ntnu.no}}}%
\hspace*{\stretch{1}}%
%\quad\href{https://orcid.org/0000-0002-6070-0784}{\protect\includegraphics[scale=0.16]{orcid_32x32.png}\textsc{orcid}:0000-0002-6070-0784}%
}

\date{Draft of \today\ (first drafted \firstdraft)}
%\date{\firstpublished; updated \updated}

%@@@@@@@@@@ new macros @@@@@@@@@@
% Common ones - uncomment as needed
%\providecommand{\nequiv}{\not\equiv}
%\providecommand{\coloneqq}{\mathrel{\mathop:}=}
%\providecommand{\eqqcolon}{=\mathrel{\mathop:}}
%\providecommand{\varprod}{\prod}
\newcommand*{\de}{\partialup}%partial diff
\newcommand*{\pu}{\piup}%constant pi
\newcommand*{\delt}{\deltaup}%Kronecker, Dirac
%\newcommand*{\eps}{\varepsilonup}%Levi-Civita, Heaviside
%\newcommand*{\riem}{\zetaup}%Riemann zeta
%\providecommand{\degree}{\textdegree}% degree
%\newcommand*{\celsius}{\textcelsius}% degree Celsius
%\newcommand*{\micro}{\textmu}% degree Celsius
\newcommand*{\I}{\mathrm{i}}%imaginary unit
\newcommand*{\e}{\mathrm{e}}%Neper
\newcommand*{\di}{\mathrm{d}}%differential
%\newcommand*{\Di}{\mathrm{D}}%capital differential
%\newcommand*{\planckc}{\hslash}
%\newcommand*{\avogn}{N_{\textrm{A}}}
%\newcommand*{\NN}{\bm{\mathrm{N}}}
%\newcommand*{\ZZ}{\bm{\mathrm{Z}}}
%\newcommand*{\QQ}{\bm{\mathrm{Q}}}
\newcommand*{\RR}{\bm{\mathrm{R}}}
%\newcommand*{\CC}{\bm{\mathrm{C}}}
%\newcommand*{\nabl}{\bm{\nabla}}%nabla
%\DeclareMathOperator{\lb}{lb}%base 2 log
%\DeclareMathOperator{\tr}{tr}%trace
%\DeclareMathOperator{\card}{card}%cardinality
%\DeclareMathOperator{\im}{Im}%im part
%\DeclareMathOperator{\re}{Re}%re part
%\DeclareMathOperator{\sgn}{sgn}%signum
%\DeclareMathOperator{\ent}{ent}%integer less or equal to
%\DeclareMathOperator{\Ord}{O}%same order as
%\DeclareMathOperator{\ord}{o}%lower order than
\newcommand*{\incr}{\triangle}%finite increment
\newcommand*{\defd}{\coloneqq}
\newcommand*{\defs}{\eqqcolon}
%\newcommand*{\Land}{\bigwedge}
%\newcommand*{\Lor}{\bigvee}
%\newcommand*{\lland}{\DOTSB\;\land\;}
%\newcommand*{\llor}{\DOTSB\;\lor\;}
%\newcommand*{\limplies}{\mathbin{\Rightarrow}}%implies
\newcommand*{\suchthat}{\mid}%{\mathpunct{|}}%such that (eg in sets)
%\newcommand*{\with}{\colon}%with (list of indices)
%\newcommand*{\mul}{\times}%multiplication
%\newcommand*{\inn}{\cdot}%inner product
%\newcommand*{\dotv}{\mathord{\,\cdot\,}}%variable place
%\newcommand*{\comp}{\circ}%composition of functions
%\newcommand*{\con}{\mathbin{:}}%scal prod of tensors
%\newcommand*{\equi}{\sim}%equivalent to 
\renewcommand*{\asymp}{\simeq}%equivalent to 
%\newcommand*{\corr}{\mathrel{\hat{=}}}%corresponds to
%\providecommand{\varparallel}{\ensuremath{\mathbin{/\mkern-7mu/}}}%parallel (tentative symbol)
\renewcommand*{\le}{\leqslant}%less or equal
\renewcommand*{\ge}{\geqslant}%greater or equal
%\DeclarePairedDelimiter\clcl{[}{]}
%\DeclarePairedDelimiter\clop{[}{[}
%\DeclarePairedDelimiter\opcl{]}{]}
%\DeclarePairedDelimiter\opop{]}{[}
\DeclarePairedDelimiter\abs{\lvert}{\rvert}
%\DeclarePairedDelimiter\norm{\lVert}{\rVert}
\DeclarePairedDelimiter\set{\{}{\}}
%\DeclareMathOperator{\pr}{P}%probability
\newcommand*{\pf}{\mathrm{p}}%probability
\newcommand*{\p}{\mathrm{P}}%probability
\newcommand*{\E}{\mathrm{E}}
\renewcommand*{\|}{\nonscript\,\vert\nonscript\;\mathopen{}}
%\DeclarePairedDelimiterX{\cond}[2]{(}{)}{#1\nonscript\,\delimsize\vert\nonscript\;\mathopen{}#2}
%\DeclarePairedDelimiterX{\condt}[2]{[}{]}{#1\nonscript\,\delimsize\vert\nonscript\;\mathopen{}#2}
%\DeclarePairedDelimiterX{\conds}[2]{\{}{\}}{#1\nonscript\,\delimsize\vert\nonscript\;\mathopen{}#2}
%\newcommand*{\+}{\lor}
%\renewcommand{\*}{\land}
\newcommand*{\sect}{\S}% Sect.~
\newcommand*{\sects}{\S\S}% Sect.~
\newcommand*{\chap}{ch.}%
\newcommand*{\chaps}{chs}%
\newcommand*{\bref}{ref.}%
\newcommand*{\brefs}{refs}%
%\newcommand*{\fn}{fn}%
\newcommand*{\eqn}{eq.}%
\newcommand*{\eqns}{eqs}%
\newcommand*{\fig}{fig.}%
\newcommand*{\figs}{figs}%
\newcommand*{\vs}{{vs}}
%\newcommand*{\etc}{{etc.}}
%\newcommand*{\ie}{{i.e.}}
%\newcommand*{\ca}{{c.}}
%\newcommand*{\eg}{{e.g.}}
\newcommand*{\foll}{{ff.}}
%\newcommand*{\viz}{{viz}}
\newcommand*{\cf}{{cf.}}
%\newcommand*{\Cf}{{Cf.}}
%\newcommand*{\vd}{{v.}}
\newcommand*{\etal}{{et al.}}
%\newcommand*{\etsim}{{et sim.}}
%\newcommand*{\ibid}{{ibid.}}
%\newcommand*{\sic}{{sic}}
%\newcommand*{\id}{\mathte{I}}%id matrix
%\newcommand*{\nbd}{\nobreakdash}%
%\newcommand*{\bd}{\hspace{0pt}}%
%\def\hy{-\penalty0\hskip0pt\relax}
%\newcommand*{\labelbis}[1]{\tag*{(\ref{#1})$_\text{r}$}}
%\newcommand*{\mathbox}[2][.8]{\parbox[t]{#1\columnwidth}{#2}}
%\newcommand*{\zerob}[1]{\makebox[0pt][l]{#1}}
\newcommand*{\tprod}{\mathop{\textstyle\prod}\nolimits}
\newcommand*{\tsum}{\mathop{\textstyle\sum}\nolimits}
%\newcommand*{\tint}{\begingroup\textstyle\int\endgroup\nolimits}
%\newcommand*{\tland}{\mathop{\textstyle\bigwedge}\nolimits}
%\newcommand*{\tlor}{\mathop{\textstyle\bigvee}\nolimits}
%\newcommand*{\sprod}{\mathop{\textstyle\prod}}
%\newcommand*{\ssum}{\mathop{\textstyle\sum}}
%\newcommand*{\sint}{\begingroup\textstyle\int\endgroup}
%\newcommand*{\sland}{\mathop{\textstyle\bigwedge}}
%\newcommand*{\slor}{\mathop{\textstyle\bigvee}}
%\newcommand*{\T}{^\intercal}%transpose
%%\newcommand*{\QEM}%{\textnormal{$\Box$}}%{\ding{167}}
%\newcommand*{\qem}{\leavevmode\unskip\penalty9999 \hbox{}\nobreak\hfill
%\quad\hbox{\QEM}}

\definecolor{notecolour}{RGB}{68,170,153}
\newcommand*{\puzzle}{{\fontencoding{U}\fontfamily{fontawesometwo}\selectfont\symbol{225}}}
%\newcommand*{\puzzle}{\maltese}
\newcommand{\mynote}[1]{ {\color{notecolour}\puzzle\ #1}}
\newcommand*{\widebar}[1]{{\mkern1.5mu\skew{2}\overline{\mkern-1.5mu#1\mkern-1.5mu}\mkern 1.5mu}}

% \newcommand{\explanation}[4][t]{%\setlength{\tabcolsep}{-1ex}
% %\smash{
% \begin{tabular}[#1]{c}#2\\[0.5\jot]\rule{1pt}{#3}\\#4\end{tabular}}%}
% \newcommand*{\ptext}[1]{\text{\small #1}}
%@@@@ Custom macros for this file @@@@
%\DeclareMathOperator*{\argsup}{arg\,sup}
\newcommand*{\dob}{degree of belief}
\newcommand*{\dobs}{degrees of belief}
\newcommand*{\yI}{\varIota}
\newcommand*{\ys}{\bm{s}}
\newcommand*{\yS}{\bm{S}}
\newcommand*{\ydt}{\incr t}
\newcommand*{\yt}{\tau}
\newcommand*{\yIme}{\varIota_{\text{ME}}}
\newcommand*{\ym}{\mathte{m}}
%@@@@@@@@@@ new macros end @@@@@@@@@@

\firmlists
\begin{document}
\captiondelim{\quad}\captionnamefont{\footnotesize}\captiontitlefont{\footnotesize}
\selectlanguage{british}\frenchspacing

%%% Title and abstract %%%
\maketitle
\abstractrunin
\abslabeldelim{}
\renewcommand*{\abstractname}{}
\setlength{\absleftindent}{0pt}
\setlength{\absrightindent}{0pt}
\setlength{\abstitleskip}{-\absparindent}
\begin{abstract}\labelsep 0pt%
  \noindent An analysis of the problem of inferring the state of a
  population of neurons from that of a sample.
\\\noindent\emph{\footnotesize Note: Dear Reader
    \amp\ Peer, this manuscript is being peer-reviewed by you. Thank you.}
% \par%\\[\jot]
% \noindent
% {\footnotesize PACS: ***}\qquad%
% {\footnotesize MSC: ***}%
%\qquad{\footnotesize Keywords: ***}
\end{abstract}

\selectlanguage{british}\frenchspacing
% \asudedication{\small ***}
% \vspace{\bigskipamount}

% \setlength{\epigraphwidth}{.7\columnwidth}
% %\epigraphposition{flushright}
% \epigraphtextposition{flushright}
% %\epigraphsourceposition{flushright}
% \epigraphfontsize{\footnotesize}
% \setlength{\epigraphrule}{0pt}
% %\setlength{\beforeepigraphskip}{0pt}
% %\setlength{\afterepigraphskip}{0pt}
% \epigraph{\emph{text}}{source}



\section{Intro}
\label{sec:intro}



The probabilistic study of the activity of networks of neurons has enjoyed
many mathematical advances from other scientific fields, statistical
mechanics in particular. Together with their mathematical understanding,
these fields have also brought into neuroscience their technical
terminology and, behind it, specific ways of physically picturing the
problem.

Although the mathematical tools are a great help, there's the risk that the
very specialized technical jargon and physical picture of other fields may
in the long run be a hinder to understanding neuronal-network activity,
because the physical phenomena at its core are very different from those of
these fields. The history of science reminds us that such semantic
distinctions are important and have often originated scientific
revolutions.\footnote{The mathematical Ptolemaic system, for example, had
  great predictive power and could be improved without limit. The reason
  for its ultimate dismissal in favour of the
  Copernican-Keplerian-Newtonian one was that the latter was based on a
  more fruitful physical \emph{picture}, which in turn led to more agile
  mathematics \citep[\chap~***]{kline1980_r1982}. Another example is semantic
  change about the idea of pressure -- pressure as something we can speak
  about even when there are no physical walls -- offered by Euler
  \citep{euler1753_r1757,truesdell1954d}[\chap~IV]{truesdell1968}, which
  eventually led to continuum thermomechanics. The most famous example is
  Einstein's \citey{einstein1905c} semantic analysis of the concept of
  \enquote{simultaneity}, which led to a completely different understanding
  of the mathematically already known Lorentz transformations and their
  related phenomena of length contraction and time dilation \citep[and of
  the relation between mass and energy, already pointed out
  by][p.~487]{poincare1900}. A final example is Born's \citey[p.~865,
  footnote]{born1926} reinterpretation of the wave function as a
  probability density rather than as charge density, which lead to a better
  understanding of the quantum-mechanical mathematical formalism.}



\section{Given sample state, infer population state}
\label{sec:sample_population}


The problem we want to consider is the inference about the state of a
population of neurons from the observation of the state of a sample of that
population. By inference we mean \emph{the numerical evaluation of our
  \dob}, in this case about the population's state. The state is taken to
be the binarized activity from a time-binned sequence. Denote by $N$ the
size of the population, by $n$ that of the sample, by $S_i(t)$ the activity
of the $i$th neuron in the population at time $t$, by
$\yS(t) \defd \bigl(S_1(t), \dotsc, S_N(t)\bigr)$ the joint activity -- the
state -- of the population, and by $s_j(t)$,
$\ys(t) \defd \bigl(s_1(t), \dotsc, s_n(t)\bigr)$ the corresponding
activities and state of the neurons making up the sample. We will discuss
the exact relationship between $\yS$ and $\ys$ in the next sections. Our
main task is to calculate our \dob
\begin{equation}
  \label{eq:main_dob_sample2pop}
  \pf[\yS(t) \| \ys(t), \yI],
\end{equation}
where $\yI$ denotes other initial information and assumptions. Many points
of our discussion apply to more general definitions of \enquote{state}.

In order to better understand what our inference is about, let's also
stress what it is \emph{not} about. Our inference is not about the
\emph{dynamics} of the population. This latter inference is roughly as
follows. We assume that the state $\yS(t)$ at time $t$ is determined
through a dynamical law by the states $\set{\yS(\yt)}_{\yt<t}$ at some
previous times together with some external quantities $Q(t)$ (such as
physical states of synapses, inputs from peripheral nervous system, and
similar extra-neuronal quantities):
\begin{equation}
  \label{eq:determ_dynamics}
  \yS(t) = F[\set{\yS(\yt)}_{\yt<t}, Q(t)].
\end{equation}
We are uncertain about the mathematical form of the dynamical law $F$ and
the values of the external quantities $Q(t)$. We can therefore consider
various \dobs: for example the one about $\yS(t)$ given only knowledge
about some previous states:
\begin{equation}
  \label{eq:prob_dynamic}
  \pf[\yS(t) \| \set{\yS(\yt)}_{\yt<t}, \yI],
\end{equation}
or the one about the dynamical law, given a time sequence of states:
\begin{equation}
  \label{eq:prob_dynamical_law}
  \pf[F \| \set{\yS(\yt)}, \yI].
\end{equation}
Our present problem doesn't concern this kind of inferences, but it's very
relevant to them: to infer the dynamics,
\eqn~\eqref{eq:prob_dynamical_law}, we usually must first infer the states
$\yS(\yt)$ from the observation of a population sample.

Note that if the dynamics is excluded from our problem, then samples at
times $\yt< t$ cannot be used for the inference of the population state at
time $t$, because such inferential chain involves the dynamics:
schematically, the inference would be
$\ys(\yt) \rightsquigarrow \yS(\yt) \rightsquigarrow \yS(t)$, and the
latter step involves the \dob~\eqref{eq:prob_dynamic}. Our discussion will
therefore refer to one time $t$ only, conveniently suppressed from
our notation.


\bigskip

To calculate our \dob~\eqref{eq:main_dob_sample2pop}, the probability
calculus requires us to specify: 1. our initial belief distribution about
the population state:
\begin{equation}
  \label{eq:initial_dob_pop}
  \pf(\yS \| \yI);
\end{equation}
2. our belief distribution about the sample state given the population
state: \textcolor{white}{(If you find this you can claim a postcard from
  me)}
\begin{equation}
  \label{eq:sample_dob}
  \pf(\ys \| \yS, \yI),
\end{equation}
which we can call the \enquote{sampling distribution}. The two distribution
above yield the distribution~\eqref{eq:main_dob_sample2pop} by Bayes's
theorem:
\begin{equation}
  \label{eq:main_bayes_theorem}
  \pf(\yS \| \ys, \yI) =
  \frac{\pf(\ys \| \yS, \yI)\;\pf(\yS \| \yI)}{
    \sum_{\yS}\pf(\ys \| \yS, \yI)\;\pf(\yS \| \yI)
  }.
\end{equation}

\bigskip

Let's investigate the sampling distribution~\eqref{eq:sample_dob}. First of
all we note that we may \emph{label} the $N$ neurons in an arbitrary way --
this doesn't mean that we consider them identical or indistinguishable. It
is then convenient to give the labels $1, \dotsc, n$ to the neurons we have
measured, and the remaining $N-n$ labels to the rest. Then we have the
identity
\begin{equation}
  \label{eq:label_sample}
  s_i=S_i, \quad i\in\set{1, \dotsc, n},
\end{equation}
and the sampling distribution~\eqref{eq:sample_dob} is a delta:
\begin{equation}
  \label{eq:sample_dob_delta}
  \pf(\ys \| \yS, \yI) = \prod_{i=1}^{n} \delt(s_i,S_i).
\end{equation}

\bigskip

Let's now investigate our initial belief~\eqref{eq:initial_dob_pop} about
the population state. To start with, I'd like to consider belief
distributions of a maximum-entropy form and offer a couple of comments on
them, because they seem very popular in the literature.

If the activities of the neurons are binarized, the set of all possible
population states $\set{\yS}$ is discrete, of cardinality $2^N$. The
set of all possible initial belief distributions for $\yS$ has then
dimension $2^N-1$ because of normalization. It is a simplex. % homeomorphic
% to the set of points
% \begin{equation}
%   \label{eq:simplex_in_R}
%   \varDelta_{2^N-1} \defd
%   \set{\bm{q} \in \clcl{0,1}^{2^N-1} \suchthat \sum_l q_l \le 1}
% \end{equation}
% and each point in this simplex represents a distribution of degrees of
% belief among the possible population states.
Each such distribution has moments -- for example,
$\E(S_3\;S_5\;S_8 \| \yI)$ --- with precise numerical values. A
maximum-entropy distribution is chosen by first choosing a subset of
distributions having specific values for some moments, and then selecting
the distribution having maximum Shannon entropy in this subset. Such a
distribution is unique because the fixed-moment subsets are convex and the
Shannon entropy is a convex function. A maximum-entropy distribution is
therefore identified by the moments chosen -- for example, first and second
moments -- and their numerical values. We can write this as
\begin{equation}
  \label{eq:initial_dob_me}
  \pf(\yS \| \ym, \yIme),
\end{equation}
a familiar example being
\begin{multline}
  \label{eq:initial_dob_me_example}
  \pf(\yS \| \set{m_i, m_{ij}}, \yIme) ={}\\
  \frac{1}{Z(\set{m_i, m_{ij}})}\,
  \exp\biggl[\sum_i h_i(\set{m_i, m_{ij}})\,S_i +
  \sum_{i,j}^{i<j} J_{ij}(\set{m_i, m_{ij}})\,S_i\,S_j\biggr],
\end{multline}
where $\set{m_i}$ are the $N$ first moments, $\set{m_{ij}}$ the
$\binom{N}{2}$ second moments, $Z$ is a normalization constant, and
$\set{h_i, J_{ij}}$ are specific one-one functions of the moments.

This kind of distributions can assign asymmetric \dob\ about the activities
of the neurons -- for example a higher belief that neuron $14$ is active
than that neuron $6$ is active. By keeping the \emph{kind} of moments (for
example, all first moments and some specific third moments) fixed but
choosing different numerical values for them we form a set of
distributions. If the number of moments considered is less than $2^N-1$
then this set has strictly lower dimension than the simplex.

A remark may be useful speaking of maximum-entropy distributions. With
other kinds of inference, convex mixtures of maximum-entropy distributions
are sometimes considered, which can be written as
\begin{equation}
  \label{eq:initial_dob_me_mixture}
  \pf(\yS \| \yIme) =
  \int\!\di\ym\;\pf(\yS \| \ym, \yIme)\; \pf(\ym \| \yIme).
\end{equation}
For our present inference, however, such a mixture is not meaningful
because redundant: its redundancy is clear when we consider the
distribution for $\ym$ conditional on perfect knowledge of the state $\yS$:
\begin{equation}
  \label{eq:dob_m_conditional_S}
  \pf(\ym \| \yS, \yIme) \propto
  \pf(\yS \| \ym, \yIme)\; \pf(\ym \| \yIme).
\end{equation}
This distribution is not a delta, that is, it says we're uncertain about
the values of the parameter $\ym$ -- even though the state $\yS$ is known!

The reason for this redundancy is that two different sets of weights
$\set*{\pf(\ym \| \yIme')}$, $\set*{\pf(\ym \| \yIme'')}$ for $\ym$ may
yield the same resulting distribution for $\yS$. Geometrically, the
distributions $\set{\pf(\yS \| \ym, \yIme)}_{\ym}$ are not the extreme
points of a simplex within the simplex $\set{\yS}$. Mixtures of
maximum-entropy distributions like~\eqref{eq:initial_dob_me_mixture} make
sense when we are making inferences about an \emph{unlimited} sequence of
states -- for example, a time sequence
$\bigl( \yS(t_1), \yS(t_2), \dotsc \bigr)$ -- because in this case each set
of weights for $\ym$ gives rise to a unique distribution \emph{for the
  unlimited sequence}. In fact, in this case such mixtures are called
models by sufficient statistics \citep[\sect~4.5]{bernardoetal1994_r2000}.


\bigskip

It's important to ask ourselves: in which experimental situations do
asymmetric belief distributions represent our initial state of knowledge
about the population state? Neural-recording instruments usually pick up
the sample of recorded neurons in an uncontrollable, unknown way. Our
initial \dob\ about the state of the full population is therefore
symmetric.

\mynote{Continue this discussion referring to
  \citep{portamanaetal2015,portamanaetal2018b}.}

$\textquotedblleft a$
\section{Inferences of dynamics}
\label{sec:inferences_dynamics}

\mynote{Important points:
  \begin{itemize}
  \item there cannot be \enquote{model-free} approaches
  \item prior gives the greatest contribution to posterior
  \item selection of a \enquote{model} is already a selection of class of priors
  \item parameters that \enquote{couple} two neurons need not have any
    direct biological meaning
  \item to make such a connection we'd need to start from the physics of
    the connections (typical delay times, other physical consequences of
    synaptic connections, and similar)
  \item some literature chooses parameter priors based on the data: mistake
    in methodology
  \item inferences about dynamical law aren't very meaningful if they aren't
    meant for a set of similar experiments
  \end{itemize}}



% \[
%   \begin{tikzcd}
%       M_{n,n}(\CC) \arrow{r}{R'_{**}(\Hat{U})} & M_{n,n}(\CC)
%     \\
%     L(\mathcal{H}) \arrow{r}{\Hat{U}} \arrow[swap]{d}{R_*}\arrow[swap]{u}{R'_*} & L(\mathcal{H}) \arrow{d}{R_*}\arrow{u}{R'_*} \\
%       M_{n,n}(\CC) \arrow{r}{R_{**}(\Hat{U})} & M_{n,n}(\CC)
%   \end{tikzcd}
% \]

% \[
%   \begin{tikzcd}
%       \CC^n \arrow{r}{R'_*(A)} & \CC^n
%     \\
%     \mathcal{H} \arrow{r}{A} \arrow[swap]{d}{R}\arrow[swap]{u}{R'} & \mathcal{H} \arrow{d}{R}\arrow{u}{R'} \\
%       \CC^n \arrow{r}{R_*(A)} & \CC^n
%   \end{tikzcd}
% \]


% \[
%   \begin{tikzcd}
%     \mathcal{H} \arrow{r}{A} \arrow[swap]{d}{R} & \mathcal{H} \arrow{d}{R} \\
%       \CC^n \arrow{r}{R_*(A)} & \CC^n
%   \end{tikzcd}
% \]

%%\setlength{\intextsep}{0.5ex}% with wrapfigure
%\begin{figure}[p!]%{r}{0.4\linewidth} % with wrapfigure
%  \centering\includegraphics[trim={12ex 0 18ex 0},clip,width=\linewidth]{maxent_saddle.png}\\
%\caption{***}\label{fig:comparison_a5}
%\end{figure}% exp_family_maxent.nb


\iffalse
\begin{acknowledgements}
  \ldots to Mari \amp\ Miri for continuous encouragement and affection, and
  to Buster Keaton and Saitama for filling life with awe and inspiration.
  To the developers and maintainers of \LaTeX, Emacs, AUC\TeX, Open Science
  Framework, Python, Inkscape, Sci-Hub for making a free and unfiltered
  scientific exchange possible.
%\rotatebox{15}{P}\rotatebox{5}{I}\rotatebox{-10}{P}\rotatebox{10}{\reflectbox{P}}\rotatebox{-5}{O}.
\sourceatright{\autanet}
\end{acknowledgements}
\fi


\newpage
% %\renewcommand*{\appendixpagename}{Appendix}
% %\renewcommand*{\appendixname}{Appendix}
% %\appendixpage
% \appendix


%%%%%%%%%%%%%%% BIB %%%%%%%%%%%%%%%

\defbibnote{prenote}{{\footnotesize (\enquote{de $X$} is listed under D,
    \enquote{van $X$} under V, and so on, regardless of national
    conventions.)\par}}
% \defbibnote{postnote}{\par\medskip\noindent{\footnotesize% Note:
%     \arxivp \mparcp \philscip \biorxivp}}

\printbibliography[prenote=prenote%,postnote=postnote
]


\end{document}
---------- cut text ----------------


%%% Local Variables: 
%%% mode: LaTeX
%%% TeX-PDF-mode: t
%%% TeX-master: t
%%% End: 
